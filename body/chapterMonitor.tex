\part{集群监控篇}

本章介绍几种集群监控的解决方案。

\chapter{Zabbix}

\section{Zabbix简介}

\section{Zabbix监控方案介绍}

\section{集群概述}

集群是一组协同工作的服务实体,用以提供比单一服务实体更具扩展性和可用性
的服务平台。

在客户端看来,一个集群就是一个完整不可细分的实体,但事实上一个集群实体
是由完成不同任务的服务节点个体所组成的。

集群实体的可扩展性是指,在集群运行中新的服务节点可以动态的加入集群实体
从而提升集群实体的综合性能。

集群实体的高可用性是指,集群实体通过其内部的服务节点的冗余使客户端免于
OUT OF SERVICE错误。简单的说,在集群中同一服务可以由多个服务节点提供,
当部分服务节点失效后,其他服务节点可以接管服务。

集群实体地址是指客户端访问集群实体获得服务资源的唯一入口地址。

负载均衡是指集群中的分发设备(服务)将用户的请求任务比较均衡(不是平均)
分发到集群实体中的服务节点计算、存储和网络资源中。一般我们将提供负载均
衡分发的设备叫做负载均衡器。负载均衡器一般具备如下三个功能:

\begin{enumerate}[itemsep=0pt,parsep=0pt]
\item 维护集群地址
\item 负责管理各个服务节点的加入和退出
\item 集群地址向内部服务节点地址的转换
\end{enumerate}

错误恢复是指集群中某个节点或某些服务节点(设备)不能正常工作(或提供服
  务),其他类似服务节点(设备)可以资源透明和持续的完成原有任务。具备
错误恢复能力是集群实体高可用性的必要条件。

负载均衡和错误恢复都需要集群实体中各个服务节点中有执行同一任务的资源存
在,而且对于同一任务的各个资源来说,执行任务所学的信息视图必须一致。

%%% Local Variables:
%%% mode: latex
%%% TeX-master: t
%%% End:
