\section{DHCP}

随着网络规模的不断扩大和网络复杂度的提高,计算机的数量经常超过可分配
的IP地址数量。同时随着便携机及无线网络的广泛使用,计算机的位置也经常变
化,相应的IP地址也必须经常更新,从而导致网络配置越来越复
杂。DHCP(Dynamic Host Configuration Protocol,动态主机配置协议)就是为
满足这些需求而发展起来的。

DHCP提供安全、可靠且简单的TCP/IP网络设置,避免了TCP/IP网络中地址的冲突,
同时也降低了管理IP地址设置的工作强度。使用DHCP主要的好处有以下几点:

\begin{enumerate}[itemsep=0pt,parsep=0pt]
\item 减少管理员的工作量
\item 减小输入错误的可能
\item 避免IP冲突
\item 当罗更改IP地址段时,不需要重新配置每台计算机的IP
\item 计算机移动不必重新配置TCP/IP信息
\item 提高了IP地址的利用率
\end{enumerate}

当一台DHCP客户端启动时,该客户端将在网络中请求IP地址,当DHCP服务器收到
申请IP地址请求后,它将从可用的地址中选择一个提供给DHCP客户端。

客户端除了可以从DHCP服务器得到IP地址信息外,还可以得到子网掩码、默认网
关、DNS服务器等其他TCP/IP信息。一般这些信息并非永久使用,而是有一个使用
的期限,这个期限被称为租约。所以DHCP服务器与客户端之间的通信分为两类,
即租约产生及租约更新。

DHCP的作用是为局域网中的每台计算机自动分配TCP/IP信息,包括IP地址、子网
掩码、网关以及DNS服务器等。其优点是终端无须配置,且网络维护方便。

动态主机配置协议(DHCP)\index{DHCP},

\begin{enumerate}[itemsep=0pt,parsep=0pt]
\item 服务端端口67/UDP
\item 客户端端口68/UDP
\item 客户端发送DHCPDISCOVER在网络中寻求地址分配
\item 服务端回应DHCPDISCOVER请求
\item 客户端发送DHCPREQUEST
\item 服务端发送DHCPACK
\item 客户端得到地址
\end{enumerate}

主配 cp /usr/share/doc/dhcp-<version>/dhcpd.conf.sample /etc/dhcpd.conf
包含以下7项 服务器即可搭建

\small{
\begin{verbatim}
1. ddns-update-style
2. subnet
3. option
4. range dynamic-bootp
5. default-lease-time
6. max-lease-time
7. host
\end{verbatim}
}
\normalsize

\small{
\begin{verbatim}
ddns-update-style interim;
ignore client-updates;
default-lease-time 600;
max-lease-time 7200;

subnet 192.168.0.0 netmask 255.255.255.0 {
    next-server 192.168.0.128;
    filename="pxelinux.0";
    option routers 192.168.0.128;
    option subnet-mask 255.255.255.0;
    option nis-domain "lavenliu.com";
    option domain-name "lavenliu.com";
    option ntp-servers 192.168.0.254;
    range dynamic-bootp 192.168.0.100 192.168.0.130;
    host stu1 {
      next-server lavenliu.com;
      hardware Ethernet 12:34:56:78:AB:CD;
      fixed-address 192.168.0.2;
    }
}
\end{verbatim}
}
\normalsize

%%% Local Variables:
%%% mode: latex
%%% TeX-master: t
%%% End:
