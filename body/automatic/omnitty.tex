\chapter{Omnitty}
\label{chap:Omnitty}

Omnitty是一个轻量级的批量运维管理工具,它是由德国人开发的。该工具目前最多支持同时操作256台远端机器,默认使用SSH协议进行通信\footnote{只有SSH可供使用,使用Omnitty机器上的SSH客户端工具。}。它的安装很简单,下面给出安装步骤:

\begin{verbatim}
# cd /usr/local/src
# tar -zxf rote-0.2.8.tar.gz
# cd rote-0.2.8
# ./configure
# make
# make install
# cd ..

# cd /usr/local/src
# tar -zxf omnitty-0.3.0.tar.gz
# cd omnitty-0.3.0
# ./configure
# make
# make install

# omnitty 
omnitty: error while loading shared libraries: librote.so.0: cannot open shared object file: No such file or directory

# cd /usr/local/src/rote-0.2.8
# cp librote.so.0.2.8 /usr/lib64/
# cd /usr/lib64
# ln -s librote.so.0.2.8 librote.so.0
\end{verbatim}

\section{Omnitty的简单使用}
\label{sec:omnittyBasicUseage}

\subsection{添加机器}
\label{subsec:addMachine}

有两种方法可以向Omnitty里添加机器,一种是一台一台的进行手工添加;另一种就是使用文件的方式一次性的添加。如果机器数量较少,我们完全可以通过手工的方式进行一台一台的添加,这样没有任何什么问题。如果要操作的机器数量大\footnote{大于10台},我们就可以通过使用文件的方式进行添加。我们把要操作的机器的IP或主机名\footnote{如果写主机名,要确保Omnitty主机可以解析。}写入到一个文件中,然后Omnitty通过打开这个文件以达到加载机器的目的。

\subsection{激活机器}
\label{subsec:activeMachine}

在上面的小节中,我们把机器已经加载进来了,这时还不能使用。因为它们未被激活,显示的颜色是灰色的。如果要激活当前光标处的机器,可以按F4键即可激活该机器。只有激活的机器才响应我们的键盘操作。

\subsection{开始操作}
\label{subsec:beginOps}

%%% Local Variables:
%%% mode: latex
%%% TeX-master: t
%%% End:
