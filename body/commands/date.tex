
\section{date命令的使用}
\label{sec:dateCmd}

\index{date}

\small{
\begin{verbatim}
# 设日期
date -s 20091112                     

# 设时间
date -s 18:30:50                     

# 7天前日期
date -d "7 days ago" +%Y%m%d         

# 5分钟前时间
date -d "5 minute ago" +%H:%M        

# 一个月前
date -d "1 month ago" +%Y%m%d        

# 日期格式转换
date +%Y-%m-%d -d '20110902'         

# 日期和时间
date +%Y-%m-%d_%X                    

# 纳秒
date +%N                             

# 换算成秒计算(1970年至今的秒数)
date -d "2012-08-13 14:00:23" +%s    

# 将时间戳换算成日期
date -d "@1363867952" +%Y-%m-%d-%T   

# 将时间戳换算成日期
date -d "1970-01-01 UTC 1363867952 seconds" +%Y-%m-%d-%T  

# 格式化系统启动时间(多少秒前)
date -d "`awk -F. '{print $1}' /proc/uptime` second ago" +"%Y-%m-%d %H:%M:%S"    
\end{verbatim}
}
\normalsize