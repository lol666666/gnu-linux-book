
\section{echo与终端颜色}
\label{sec:echoCmd}

echo\index{echo}会将输入的字符串送往标准输出。输出的字符串间以空白字符隔开, 并在最
后加上换行号。

参数:

\begin{enumerate}[itemsep=0pt,parsep=0pt]
\item \-n 不要在最后自动换行 
\item \-e 若字符串中出现以下字符,则特别加以处理,而不会将它当成一般文字输出: 
\begin{verbatim}
\a 发出警告声; 
\b ***前一个字符; 
\c 最后不加上换行符号; 
\f 换行但光标仍旧停留在原来的位置; 
\n 换行且光标移至行首; 
\r 光标移至行首,但不换行; 
\t 插入tab; 
\v 与\f相同; 
\\ 插入\字符; 
\nnn 插入nnn(八进制)所代表的ASCII字符; 
–help 显示帮助 
–version 显示版本信息
\end{verbatim}
\end{enumerate}

\subsection{终端颜色}

echo字体颜色和背景颜色 

-e enable interpretation of the backslash-escaped characters listed below 

字背景颜色范围:40-47

\begin{table}[!htbp]
  \centering
  \begin{tabular}{llll}
    \toprule
    R & G & B & Color \\
    \midrule
    0 & 0 & 0 & Black \\
    0 & 0 & 1 & Blue \\
    0 & 1 & 0 & Green \\
    0 & 1 & 1 & Cyan \\
    1 & 0 & 0 & Red \\
    1 & 0 & 1 & Magenta \\
    1 & 1 & 0 & Yellow \\
    1 & 1 & 1 & White \\
    \bottomrule
  \end{tabular}
  \caption{颜色表\cite{computersystem}}
  \label{tab:colorTable}
\end{table}

\begin{verbatim}
40:黑 
41:深红 
42:绿 
43:*** 
44:蓝色 
45:紫色 
46:深绿 
47:白色
\end{verbatim}

字颜色:30-37

ANSI控制码的说明:

\begin{verbatim}
\e[0m 关闭所有属性 
\e[1m 设置高亮度 
\e[4m 下划线 
\e[5m 闪烁 
\e[7m 反显 
\e[8m 消隐 
\e[30m — \e[37m 设置前景色 
\e[40m — \e[47m 设置背景色 
\e[nA 光标上移n行 
\e[nB 光标下移n行 
\e[nC 光标右移n行 
\e[nD 光标左移n行 
\e[y;xH设置光标位置 
\e[2J 清屏 
\e[K 清除从光标到行尾的内容 
\e[s 保存光标位置 
\e[u 恢复光标位置 
\e[?25l 隐藏光标 
\e[?25h 显示光标
\end{verbatim}

下面看一个例子:
\begin{verbatim}
for i in `seq 0 7` ; do echo -e "\033[30;4${i}m      \033[0m"; \ 
done
\end{verbatim}

输出结果为:
\begin{figure}[hbtp]
  \centering
  \includegraphics[width=.15\textwidth]{img/color.png}
  \caption{终端颜色效果}
  \label{fig:TermColor}
\end{figure}