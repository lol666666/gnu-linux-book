\documentclass[a4paper,11pt]{book}
\usepackage{ctex}
\begin{document}
	要想使用该工具,首先系统上要安装相应的软件包。安装完毕之后,在/opt目录下回多出一个MegaRAID/MegaCli的目录,
	
	本次使用该命令行工具的场景:
	在第4、第5个插槽,新添加了两块Intel P3700 SSD盘,
	
	1. 查看RAID卡的设备号
	\begin{verbatim}
		# cd /opt/MegaRAID/MegaCli
		# ./MegaCli64 -PDList -aAll |grep "Device ID"
	\end{verbatim}

	Enclosure Device ID: 252
	Enclosure Device ID: 252 
	Enclosure Device ID: 252
	Enclosure Device ID: 252
	Enclosure Device ID: 252
	Enclosure Device ID: 252
	说明:上面的输出,表示我们有一个RAID卡,这个卡下面有6个盘。这里的ID号需要记下来,
	后面做RAID时需要用到。
	
	2. 查看Slot ID,有没有错序的情况
	\begin{verbatim}
		# ./MegaCli64 -PDList -aAll |grep "Slot"  
	\end{verbatim}

	Slot Number: 0
	Slot Number: 1
	Slot Number: 2
	Slot Number: 3
	Slot Number: 4
	Slot Number: 5
	说明:这里的Slot Number号需要记下来,后面做RAID时需要用到。
	
	3. 查看Foreign信息
	\begin{verbatim}
		# ./MegaCli64 -PDList -aAll |grep "Foreign State"
	\end{verbatim}
	Foreign State: None
	Foreign State: None 
	Foreign State: None
	Foreign State: None
	Foreign State: Foreign \# 这是新增加的
	Foreign State: Foreign \# 这是新增加的
	
	4. 清除Foreign信息
	\begin{verbatim}
		# ./MegaCli64 -CfgForeign -Clear -a0
	\end{verbatim}
	
	5. 重新做RAID,在Slot4和Slot5上做RAID0
	\begin{verbatim}
		# ./MegaCli64 -CfgLdAdd -r0 [252:4,252:5] WT Direct -a0
	\end{verbatim}
	说明:如果做RAID1,只需要把r0改为r1即可。
	
	附录:名词解释
	磁盘缓存策略:
	WT(Write Through)
	WB(Write Back)
	NORA(NO Read Ahead)
	RA(Read Ahead)
	ADRA(Adaptive Read Ahead)
\end{document}
