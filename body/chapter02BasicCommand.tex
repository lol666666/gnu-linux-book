\chapter{常用命令使用}
\label{sec:BasicCommand}

\input{body/commands/shortcuts}

\section{使用man page获得帮助}
\label{sec:getHelp}

当我们遇到不会用的系统命令时,该怎么办呢?或许你第一个想到的
是百度或谷歌,这样想很正常,可以节省很多时间。如果每次遇到不会的命令,
都去找网络,个人觉得这不是一件好的事情,不利于我们自身的提高。
在开源界遇到了问题,有一句口号是:“有问题,找男人\footnote{并不是真男人,而是man page}”。

如果不依靠互联网,该怎么解决呢?那就是依赖系统自带的man page\index{man
  page}了。通过它我们可以获取绝大部分的系统帮助信息。



\section{echo与终端颜色}
\label{sec:echoCmd}

echo\index{echo}会将输入的字符串送往标准输出。输出的字符串间以空白字符隔开, 并在最
后加上换行号。

参数:

\begin{enumerate}[itemsep=0pt,parsep=0pt]
\item \-n 不要在最后自动换行 
\item \-e 若字符串中出现以下字符,则特别加以处理,而不会将它当成一般文字输出: 
\begin{verbatim}
\a 发出警告声; 
\b ***前一个字符; 
\c 最后不加上换行符号; 
\f 换行但光标仍旧停留在原来的位置; 
\n 换行且光标移至行首; 
\r 光标移至行首,但不换行; 
\t 插入tab; 
\v 与\f相同; 
\\ 插入\字符; 
\nnn 插入nnn(八进制)所代表的ASCII字符; 
–help 显示帮助 
–version 显示版本信息
\end{verbatim}
\end{enumerate}

\subsection{终端颜色}

echo字体颜色和背景颜色 

-e enable interpretation of the backslash-escaped characters listed below 

字背景颜色范围:40-47

\begin{table}[!htbp]
  \centering
  \begin{tabular}{llll}
    \toprule
    R & G & B & Color \\
    \midrule
    0 & 0 & 0 & Black \\
    0 & 0 & 1 & Blue \\
    0 & 1 & 0 & Green \\
    0 & 1 & 1 & Cyan \\
    1 & 0 & 0 & Red \\
    1 & 0 & 1 & Magenta \\
    1 & 1 & 0 & Yellow \\
    1 & 1 & 1 & White \\
    \bottomrule
  \end{tabular}
  \caption{颜色表\cite{computersystem}}
  \label{tab:colorTable}
\end{table}

\begin{verbatim}
40:黑 
41:深红 
42:绿 
43:*** 
44:蓝色 
45:紫色 
46:深绿 
47:白色
\end{verbatim}

字颜色:30-37

ANSI控制码的说明:

\begin{verbatim}
\e[0m 关闭所有属性 
\e[1m 设置高亮度 
\e[4m 下划线 
\e[5m 闪烁 
\e[7m 反显 
\e[8m 消隐 
\e[30m — \e[37m 设置前景色 
\e[40m — \e[47m 设置背景色 
\e[nA 光标上移n行 
\e[nB 光标下移n行 
\e[nC 光标右移n行 
\e[nD 光标左移n行 
\e[y;xH设置光标位置 
\e[2J 清屏 
\e[K 清除从光标到行尾的内容 
\e[s 保存光标位置 
\e[u 恢复光标位置 
\e[?25l 隐藏光标 
\e[?25h 显示光标
\end{verbatim}

下面看一个例子:
\begin{verbatim}
for i in `seq 0 7` ; do echo -e "\033[30;4${i}m      \033[0m"; \ 
done
\end{verbatim}

输出结果为:
\begin{figure}[hbtp]
  \centering
  \includegraphics[width=.15\textwidth]{img/color.png}
  \caption{终端颜色效果}
  \label{fig:TermColor}
\end{figure}


\section{date命令的使用}
\label{sec:dateCmd}

\index{date}

\small{
\begin{verbatim}
# 设日期
date -s 20091112                     

# 设时间
date -s 18:30:50                     

# 7天前日期
date -d "7 days ago" +%Y%m%d         

# 5分钟前时间
date -d "5 minute ago" +%H:%M        

# 一个月前
date -d "1 month ago" +%Y%m%d        

# 日期格式转换
date +%Y-%m-%d -d '20110902'         

# 日期和时间
date +%Y-%m-%d_%X                    

# 纳秒
date +%N                             

# 换算成秒计算(1970年至今的秒数)
date -d "2012-08-13 14:00:23" +%s    

# 将时间戳换算成日期
date -d "@1363867952" +%Y-%m-%d-%T   

# 将时间戳换算成日期
date -d "1970-01-01 UTC 1363867952 seconds" +%Y-%m-%d-%T  

# 格式化系统启动时间(多少秒前)
date -d "`awk -F. '{print $1}' /proc/uptime` second ago" +"%Y-%m-%d %H:%M:%S"    
\end{verbatim}
}
\normalsize

\section{yum命令的使用}
\label{sec:yumCmd}
\index{yum}
安装好系统时,在/etc/yum.repos.d目录下回有一个rhel-debuginfo.repo的文件,
我们这里以redhat系统为例进行讲解。不管这个配置文件的名字如何,但文件的
扩展名须为.repo,如redhat.repo也是可以的。我们需要做一些准备工作。

准备系统镜像文件并挂载到本地:

\small{
\begin{verbatim}
[root@iLiuc ~]# ls 
rhel-server-5.5-i386-dvd.iso

[root@iLiuc ~]# mount -o loop rhel-server-5.5-i386-dvd.iso /media
\end{verbatim}
}
\normalsize

复制镜像里的文件到本地目录:

\begin{verbatim}
[root@iLiuc ~]# mkdir /iso
[root@iLiuc ~]# cp -r /media/* /iso
\end{verbatim}

修改这个配置文件:

\begin{verbatim}
[root@iLiuc ~]# cat /etc/yum.repos.d/rhel-debuginfo.repo
[rhel-debuginfo]
name=Red Hat Enterprise Linux $releasever - $basearch - Debug
baseurl=file:///iso/Server
enabled=1
gpgcheck=0
\end{verbatim}

几点说明:

\begin{quote}
    1. [rhel-debuginfo] 中括号里的内容可以随意写 \\
    2. name 这一行可有可无 \\
    3. baseurl 这行要指定我们的资源在哪里 \\
    4. file:// 说明我们使用什么协议,也可以是ftp://等 \\
    5. /iso/Server 指明我们的源在 /iso/Server 目录下
\end{quote}

配置好之后,如何使用呢?直接看操作吧:

\begin{enumerate}[itemsep=0pt,parsep=0pt]
\item 列出我们有哪些yum仓库
  \small{
\begin{verbatim}
[root@iLiuc ~]# yum repolist
\end{verbatim}
  }
  \normalsize

\item 列出仓库里的包
\begin{verbatim}
[root@iLiuc ~]# yum list
Deployment_Guide-en-US.noarch    5.2-11               installed     
GConf2.i386                      2.14.0-9.el5         installed     
ImageMagick.i386                 6.2.8.0-4.el5_1.1    installed     
MAKEDEV.i386                     3.23-1.2             installed     
NetworkManager.i386              1:0.7.0-10.el5       installed     
NetworkManager-glib.i386         1:0.7.0-10.el5       installed     
NetworkManager-gnome.i386        1:0.7.0-10.el5       installed     
ORBit2.i386                      2.14.3-5.el5         installed    
....
yum-utils.noarch                 1.1.16-13.el5        rhel-debuginfo
yum-verify.noarch                1.16-13.el5          rhel-debuginfo
yum-versionlock.noarch           1.1.16-13.el5        rhel-debuginfo
zisofs-tools.i386                1.0.6-3.2.2          rhel-debuginfo
zsh.i386                         4.2.6-3.el5          rhel-debuginfo
zsh-html.i386                    4.2.6-3.el5          rhel-debuginfo
\end{verbatim}
\end{enumerate}

一些说明:
\begin{quote}
    1. 第一列是我们的软件包名 \\
    2. 第二列是对应软件包的版本号 \\
    3. 第三列 \\
    + installed表明该软件包已安装 \\
    + rhel-debuginfo表明包未安装 
\end{quote}

几个常用的yum命令:
\begin{table}[!htbp]
  \centering
    \caption{yum常用命令选项}
    \begin{tabular}{ll}
      \toprule
      命令           & 说明 \\
      \midrule
      repolist       & 列出我们有哪些yum仓库 \\
      list           & 列出仓库里有哪些软件包 \\
      install        & 安装软件包的命令 \\
      groupinstall   & 安装软件包组 \\
      erase          & 移除一个或多个软件包 \\
      \bottomrule
    \end{tabular}
\end{table}

\subsection{一些实例}
\label{sec:yumExamples}

% \input{body/chapter02_zypper}


\section{parted命令的使用}
\label{sec:PartedCmd}

Gnu/Linux系统的分区工具通常可以使用fdisk与parted。我们用的比较多的工具
就是fdisk了,这里不介绍它的使用了。这里简单的介绍如何使用parted工具,对
于分区表通常有MBR分区表和GPT分区表对于磁盘大小小于2T的磁盘,我们可以使
用fdisk和parted命令工具进行分区对于MBR分区表的特点(通常使用fdisk命令进
行分区)所支持的最大磁盘大小:2T最多支持4个主分区或者是3个主分区加上一
个扩展分区对于GPT分区表的特点(使用parted命令进行分区)支持最大
卷:18EB(1EB=1024TB)最多支持128个分区。

对于parted命令工具分区的介绍

最后,fdisk与parted有些差异。fdisk分区完毕后,需要使用“w”命令才能保存
之前所做的一些操作;而parted则是实时的,每一步操作不需要保存,即时生
效。


\section{mount命令的使用}
\label{sec:mountCmd}

如何挂载iso镜像文件呢?我们可以使用一下mount\index{mount}命令:

\small{
\begin{verbatim}
[root@iLiuc ~]# mount -o loop rhel-server-5.5-i386-dvd.iso /mnt
意思是把挂rhel-server-5.5-i386-dvd.iso载到/mnt目录下,不过
你得事先有这个镜像文件
\end{verbatim}
}
\normalsize


\input{body/commands/grep}

\section{crontab命令的使用}
\label{sec:crontabCmd}

假想这样一个场景:每天的凌晨两点,领导都会要求你你重启服务器(当然,这
有点变态)。这时,你该怎么办?你是不是每天凌晨两点都要从温暖的被窝里爬
出来,然后远程连接服务器,然后重启服务器,然后重新钻进被窝,然后失眠
了...。每天都如此,我想你一定会奔溃的。

crontab\index{crontab}命令可以解救你!crontab几个字段的说明:

\small{
\begin{verbatim}
  field          allowed values
  -----          --------------
  minute         0-59
  hour           0-23
  day of month   1-31
  month          1-12 (or names, see below)
  day of week    0-7 (0 or 7 is Sun, or use names)
\end{verbatim}
}
\normalsize

\small{
\begin{verbatim}
  # 查看当前用户的crontab
  [root@iLiuc ~]# crontab -l
  */2 * * * * /usr/lib/clear-server/cleargard/cleargard.sh
  上面语句的意思是,每2分钟去执行/usr/lib/clear-server/cleargard目录下的
  cleargard.sh脚本,只要系统一直运行,它就会循环往复的执行。

  # 编辑crontab
  [root@iLiuc ~]# crontab -e
\end{verbatim}
}
\normalsize


\input{body/commands/find}


\section{top命令的使用}
\label{sec:topCmd}

top\index{top}命令可动态显示服务器的进程信息,用户可以通过按键来刷新当
前状态。别的不多说,给个例子看看:

\begin{verbatim}
Tasks: 202 total,   2 running, 199 sleeping,   0 stopped,   1 zombie
Cpu(s):  7.9%us,  1.9%sy,  0.0%ni, 89.5%id,  0.7%wa,  0.0%hi,  0.0%si,  0.0%st
Mem:   6003152k total,  1909420k used,  4093732k free,    73688k buffers
Swap:  6180860k total,        0k used,  6180860k free,   893544k cached

  PID USER      PR  NI  VIRT  RES  SHR S %CPU %MEM    TIME+  COMMAND                                                   
 3852 richard   20   0 1329m 127m  31m S   22  2.2   2:11.41 vlc                                                                 
 2891 richard    9 -11  418m 6988 4660 S    6  0.1   0:37.86 pulseaudio 
 2880 richard   20   0 1669m  90m  35m S    5  1.6   1:35.32 gnome-shell
 2452 root      20   0  303m  74m  61m S    4  1.3   1:23.90 Xorg
 3051 richard   20   0  872m 200m  52m S    1  3.4   1:32.25 firefox
   10 root      20   0     0    0    0 S    0  0.0   0:00.57 rcuos/2
   77 root      20   0     0    0    0 R    0  0.0   0:01.30 kworker/3:1
  909 root     -51   0     0    0    0 S    0  0.0   0:10.11 irq/48-iwlwifi
 3025 richard   20   0  581m  19m  11m S    0  0.3   0:01.72 gnome-terminal
 3942 richard   20   0 99.6m  15m 5028 S    0  0.3   0:02.35 python
 4025 richard   20   0 17460 1408  980 R    0  0.0   0:00.04 top
    1 root      20   0 24740 2620 1352 S    0  0.0   0:00.88 init
    2 root      20   0     0    0    0 S    0  0.0   0:00.00 kthreadd
    3 root      20   0     0    0    0 S    0  0.0   0:00.08 ksoftirqd/0
    5 root       0 -20     0    0    0 S    0  0.0   0:00.00 kworker/0:0H                                                                            
    6 root      20   0     0    0    0 S    0  0.0   0:01.68 kworker/u16:0
    7 root      20   0     0    0    0 S    0  0.0   0:01.60 rcu_sched
    8 root      20   0     0    0    0 S    0  0.0   0:01.24 rcuos/0
    9 root      20   0     0    0    0 S    0  0.0   0:00.45 rcuos/1
   11 root      20   0     0    0    0 S    0  0.0   0:00.28 rcuos/3
   12 root      20   0     0    0    0 S    0  0.0   0:00.00 rcuos/4
   13 root      20   0     0    0    0 S    0  0.0   0:00.00 rcuos/5
   14 root      20   0     0    0    0 S    0  0.0   0:00.00 rcuos/6
   15 root      20   0     0    0    0 S    0  0.0   0:00.00 rcuos/7
   16 root      20   0     0    0    0 S    0  0.0   0:00.00 rcu_bh
\end{verbatim}

\section{free命令的使用}
\label{sec:freeCmd}

\subsection{常用选项}
\label{subsec:freeOptions}

free 命令是查看系统内存的使用情况的。下面介绍几个常用的选项,

\begin{table}[htbp]
  \centering
    \caption{free常用选项}
    \label{tab:freeSomeOpts}
    \begin{tabular}{cl}
      \toprule
      选项     & 说明 \\
      \midrule
      -m        & 以 MB 为单位显示当前系统的内存使用情况 \\
      -g        & 以 GB 为单位显示当前系统的内存使用情况 \\
      \bottomrule
    \end{tabular}
\end{table}

\subsection{一些实例}
\label{subsec:freeInstances}



\section{xargs命令的使用}
\label{sec:xargsCmd}
\index{xargs}

\section{tr命令的使用}
\label{sec:trCmd}
\index{tr}

\input{body/commands/tar}

\section{read命令的使用}
\label{sec:readCmd}
\index{read}

\subsection{常用选项}

\subsection{一些实例}

\input{body/commands/cut}

\section{sort命令的使用}
\label{sec:sortCmd}
\index{sort}

\subsection{常用选项}
\label{subsec:sortOptions}

下面介绍几个常用的cut命令的选项,

\begin{table}[htbp]
  \centering
    \caption{sort常用选项}
    \label{tab:sortSomeOpts}
    \begin{tabular}{cl}
      \toprule
      选项     & 说明 \\
      \midrule
      -c        & 按照字符进行分割 \\
      -d        & 指定分割字段的分隔符,默认是tab \\
      -f        & 指定要显示的列 \\
      \bottomrule
    \end{tabular}
\end{table}

\subsection{一些实例}
\label{subsec:sortInstances}

\input{body/commands/lsof}

\section{netstat命令的使用}
\label{sec:netstatCmd}
\index{netstat}

\input{body/commands/tcpdump}

\section{traceroute命令的使用}
\label{sec:tracerouteCmd}

\section{wget命令的使用}
\label{sec:wgetCmd}

通常用来在命令行下面下载文件用的一个命令。

\section{screen命令的使用}
\label{sec:screenCmd}

运维人员经常需要SSH到远程登录Unix/Linux服务器,经常执行一些需要很长时间
才能完成的任务,比如作者最近在测试PCIe SSD卡的稳定性、InfiniBand网卡的
带宽、系统备份等等。通常,我们会为每一个任务打开一个远程终端,在执行任
务期间,必须等待命令执行完毕,表明该任务正常结束。在此期间,不能关闭终
端或断开链接,否则这个任务一起被终止。

\subsection{screen常用参数}

\subsection{使用screen}

\begin{enumerate}
\item 创建一个新窗口
\item 查看已创建的窗口
\item 会话分离与恢复
\end{enumerate}


\section{iptables的使用}
\label{sec_iptables}


\begin{verbatim}
1.1)设定INPUT为ACCEPT 
    # iptables -P INPUT ACCEPT

1.2)设定OUTPUT为ACCEPT
    # iptables -P OUTPUT ACCEPT

1.3)设定FORWARD为ACCEPT
    # iptables -P FORWARD ACCEPT


2)定制源地址访问策略

2.1)接收来自192.168.0.3的IP访问
    # iptables -A INPUT -i eth0 -s 192.168.0.3 -j ACCPET

2.2)拒绝来自192.168.0.0/24网段的访问
    # iptables -A INPUT -i eth0 -s 192.168.0.0/24 -j DROP
 

3)目标地址192.168.0.3的访问给予记录,并查看/var/log/message
    # iptables -A INPUT -s 192.168.0.3 -j LOG


4)定制端口访问策略

4.1)拒绝任何地址访问本机的111端口
    # iptables -A INPUT -i eth0 -p tcp --dport 111 -j DROP

4.2)拒绝192.168.0.0/24网段的1024-65534的源端口访问SSH
    # iptables -A INPUT -i eth0 -p tcp -s 192.168.0.0/24 \
      --sport 1024:65534 --dport ssh -j DROP

5)定制CLIENT端的防火墙访问状态

5.1)清除所有已经存在的规则;
    # iptables -F

5.2)设定预设策略,除了INPUT设为DROP,其他为ACCEPT;
    # iptables -P INPUT DROP
    # iptables -P OUTPUT ACCEPT
    # iptables -P FORWARD ACCEPT

5.3)开放本机的lo可以自由访问;
    # iptables -A INPUT -i lo -j ACCEPT

5.4)设定有相关的封包状态可以进入本机;
    # iptables -A INPUT -i eth0 -m state \
      --state RELATED,ESTABLISHED -j ACCEPT
    # iptables -A INPUT -m state --state INVALID -j DROP


6)定制防火墙的MAC地址访问策略

6.1)清除所以已经存的规则
# iptables -F
# iptables -X
# iptables -Z

6.2)将INPUT设为DROP
# iptables -P INPUT DROP

6.3)将目标计算机的MAC设为ACCEPT
# iptables -A INPUT -m mac --mac-source \
  00-C0-9F-79-E1-8A -j ACCEPT

7)设定ICMP包,状态为8的被DROP掉
  # iptables -A INPUT -i eth0 -p icmp \
    --icmp-type 8 -j DROP

8)定制防火墙的NAT访问策略

8.1)清除所有策略
# iptables -F

8.2)重置ip_forward为1
# cat "1" > /proc/sys/net/ipv4/ip_forward

8.3)通过MASQUERADE设定来源于192.168.6.0网段的IP通过192.168.6.217转发出去
# iptables -t nat -A POSTROUTING -s 192.168.6.0 -o \
  192.168.6.217 -j MASQUERADE

8.4)通过iptables观察转发的数据包
# iptables -L -nv


9)定制防火墙的NAT访问策略

9.1)清除所有NAT策略
# iptables -F -t nat

9.2)重置ip_forward为1
# echo "1" > /proc/sys/net/ipv4/ip_forward

9.3)通过SNAT设定来源于192.168.6.0网段通过eth1转发出去
# iptables -t nat -A POSTROUTING -o eth1 \
  -j SNAT --to-souce 192.168.6.217

9.4)用iptables观察转发的数据包
# iptables -L -nv

10)端口转发访问策略

10.1)清除所有NAT策略
# iptables -F -t nat

10.2)通过DNAT设定为所有访问192.168.6.217的22端口,都访问到192.168.6.191的22端口
# iptables -t nat -A PREROUTING -d 192.168.6.217 \
  -p tcp --dport 22 -j DNAT --to-destination 192.168.6.191:22

10.3)设定所有到192.168.6.191的22端口的数据包都通过FORWARD转发
# iptables -A FORWARD -p tcp -d 192.168.6.191 --dport 22 -j ACCEPT

10.4)设定回应数据包,即通过NAT的POSTROUTING设定,使通讯正常
# iptables -t nat -I POSTROUTING -p tcp --dport 22 -j MASQUERADE

============================================================
# iptables -A INPUT -m state --state NEW -p tcp --dport 25 -j ACCEPT
# iptables -A INPUT -m state --state NEW -j DROP

修改目标地址在路由之前
# iptables -t nat -A PREROUTING -d 2.2.2.2 -p tcp \
  --dport 80 -j DNAT --to-destination 192.168.1.1:80

修改源地址在路由之后
# iptables -t nat -A POSTROUTING -s 192.168.1.0/24 \
  -o eth1 -j SNAT --to-source 1.1.1.1
\end{verbatim}

\subsection{内网机器通过iptables访问互联网}
\label{sec:InternalMachineInternet}

内网客户机通过一台Gnu/Linux服务器访问互联网。服务器的eth0网卡可以访问互
联网,服务器的eth1网卡与内网客户机相连。客户机通过该服务器访问互联网。

实验环境,

\begin{table}[!htbp]
  \centering
  \caption{iptables实验环境}
  \label{tab:iptables_test}
  \begin{tabular}{llll}
    \toprule
    角色 & IP & 网卡 & 网关 \\
    \midrule
    服务端 & 10.11.1.71/24 & eth0 & 10.11.1.1 \\
           & 192.168.56.109 & eth1 & 192.168.56.1 \\
    客户端 & 192.168.56.101 & eth0 & 192.168.56.109 \\
    \bottomrule
  \end{tabular}
\end{table}

实验环境已具备,接下来配置网关服务器。只需要简单的几步就可完成内网客户
机的上网需求。首先,开启服务器的内核转发功能,使其具备路由功能,设置如
下,

\begin{verbatim}
# echo 1 > /proc/sys/net/ipv4/ip_forward
\end{verbatim}

或则,

\begin{verbatim}
# sysctl -w net.ipv4.ip_forward=1
# sysctl -p
\end{verbatim}

设置完毕,可以在内网客户机上测试与服务器的连通性,

\begin{verbatim}
[root@client ~]# ping -c 2 192.168.56.109
PING 192.168.56.109 (192.168.56.109) 56(84) bytes of data.
64 bytes from 192.168.56.109: icmp_seq=1 ttl=64 time=5.76 ms
64 bytes from 192.168.56.109: icmp_seq=2 ttl=64 time=0.303 ms

--- 192.168.56.109 ping statistics ---
2 packets transmitted, 2 received, 0% packet loss, time 1002ms
rtt min/avg/max/mdev = 0.303/3.034/5.765/2.731 ms

[root@client ~]# ping -c 2 10.11.1.71
PING 10.11.1.71 (10.11.1.71) 56(84) bytes of data.
64 bytes from 10.11.1.71: icmp_seq=1 ttl=64 time=0.331 ms
64 bytes from 10.11.1.71: icmp_seq=2 ttl=64 time=0.331 ms

--- 10.11.1.71 ping statistics ---
2 packets transmitted, 2 received, 0% packet loss, time 999ms
rtt min/avg/max/mdev = 0.331/0.331/0.331/0.000 ms
\end{verbatim}


其次,配置NAT规则。由上一步骤的配置后,我们可以ping通服务器的各个网卡
的IP地址,但是内网主机还是无法访问互联网。内网机器需要通过地址转换后,
才能访问互联网。接下来配置NAT规则,

\begin{verbatim}
[root@server ~]# iptables -t nat -A POSTROUTING -s 192.168.56.0/24 \
> -o eth0 -j SNAT --to-source 10.11.1.71
[root@server ~]# iptables -A FORWARD -i eth1 -j ACCEPT
\end{verbatim}

\subsection{iptables之端口转发}
\label{sec:iptables_port_forward}

当我们的Web服务器在局域网内部,而且没有可在Internet上使用的真实IP地址,
那就可以使用DNAT让防火墙把所有到它自己HTTP端口的包转发给局域网内部真正
的Web服务器。目的地址可以是一个范围,这样的话,DNAT会为每一个流随机分配
一个地址。

注意,DNAT只能用在nat表的PREROUTING和OUTPUT链中,或者是被两条链调用的链
里。DNAT的选项为\verb|--to-destinatiion|,一个例子为,

\begin{verbatim}
# iptables -t nat -A PREROUTING -p tcp -d 116.236.245.210 \ 
--dport 22 -j DNAT --to-destinatiion 10.10.7.153-10.10.7.158
\end{verbatim}

解释:指定要写入IP头的地址,这也是包要被转发到的地方。上面的例子就是把
所有发往地址116.236.245.210的包都转发到一段局域网使用的私有地址中,
即10.10.7.153到10.10.7.158。如前所述,在这种情况下,每个连接都会被随机
分配到一个要转发到的地址,但同一个连接流总是使用同一个地址。我们可以只
指定一个IP地址作为参数,这样所有的包都被转发到同一台机器。我们还可以在
地址后指定一个或一个范围的端口。如:\verb|--to-destinatiion 10.10.7.158:80|或\verb|--to-destinatiion 10.10.7.158:80-100|。要注意,
只有先用\verb|--protocol|指定了TCP或UDP协议,才能使用端口。

下面来一个具体的例子,大致理解一下它是如何工作的。比如,我们想通
过Internet发布我们的网站,但是HTTP服务器在我们的内网里,而且我们对外只
有一个公有IP地址,就是防火墙那个对外的IP,即INET\_IP。防火墙还有一个内
网的IP,即LAN\_IP,HTTP的IP是HTTP\_IP(这也是内网地址)。为了完成我们设
想,要做的第一件事就是把下面的这个简单的规则加入到nat表中的PREROUTING链
中:

\begin{verbatim}
# iptables -t nat -A PREROUTING --dst INET_IP \ 
-p tcp --dport 80 -j DNAT --to-destinatiion HTTP_IP
\end{verbatim}

有了这条规则,所有从Internet来的并防火墙的80端口的数据包都会被转发到内
网的HTTP服务器上。下面是数据包

\begin{figure}[hbtp]
  \centering
  \caption{kvm桥接方式的网络拓扑}
  \label{fig:kvm_bridge_network}
  \includegraphics{graph/kvm_network.pdf}
\end{figure}


\section{qperf命令的使用}
\label{sec:qperfCmd}

我们在做网络服务器的时候,通常会很关心网络的带宽和延迟。因为我们的很多
协议都是request-response协议,延迟决定了最大的QPS,而带宽决定了最大的负
荷。 通常我们知道自己的网卡是什么型号,交换机什么型号,主机之间的物理距
离是多少,理论上是知道带宽和延迟是多少的。但是现实的情况是,真正的带宽
和延迟情况会有很多变数的,比如说网卡驱动,交换机跳数,丢包率,协议栈配
置,就实际速度而言,都很大的影响了数值的估算。 所以我们需要找到工具来实
际测量下。

SUSE11sp2发行版里面自带,方便安装,专业有效,能够针对TCP和RDMA进行带宽
和延迟的详细测试。

\begin{verbatim}
# zypper install -y qperf
\end{verbatim}

由于我们需要测试Infiniband的传输速率,在安装之前请先确认安装
了InfiniBand的相关包,比如librdmacm,libibverbs等。另外,也可以选择使用
源码包编译和安装qperf,但是需要注意,在安装之前也需要将infiniband相关的
包先安装上,否则RDMA的相关测试也将无法进行。

\begin{verbatim}
# zypper install -y librdmacm libibverbs
\end{verbatim}

\subsection{参数说明及示例}

qperf分为服务器端和客户端。客户端通过发送请求并获得响应来获得服务器端和
客户端之间的网络带宽以及延迟等信息。

\begin{tabular}{lp{20em}}
  \toprule
  参数名       & 参数说明 \\
  \midrule
  <server\_ip>	& 指定服务器的地址 \\
  time            & 指定网络测试时间。默认单位为秒,单位可以通过后缀为m,h,d指定为分钟,小时,天 \\
  conf	        & 测试输出中显示本地和远端服务器和操作系统配置 \\
  use\_bits\_per\_sec & 使用b(bit)而不是B(byte)来显示网络速度 \\
  precision 2	& 设置显示小数点后几位。这里设置为显示小数点后两位 \\
  verbose\_more	& 显示更详细的配置和状态信息 \\
  loop msg\_size:1:1025k:*2 	& loop表示对指定的指标值进行轮询。这里设置为对msg\_size轮询1,2,4,8…1024k,获得对应的测试结果,下次测试的指标值是上次测试指标值的*2倍 \\
  tcp\_bw	& 对tcp的带宽进行测试 \\
  tcp\_lat	& 对tcp的延迟进行测试 \\
  udp\_bw	& 对udp的带宽进行测试 \\
  udp\_lat	& 对udp的延迟进行测试 \\
  sdp\_bw	& 对sdp的延迟进行测试 \\
  sdp\_lat	& 对sdp的延迟进行测试 \\
\bottomrule
\end{tabular}

\section{iperf命令的使用}
\label{sec:iperfCmd}

iperf工具我们主要

首先到官网获取iperf工具,并把该工具放到合适的位置。
\begin{verbatim}
# wget https://iperf.fr/download/iperf_2.0.2/iperf_2.0.2-4_amd64
# chmod +x iperf_2.0.2-4_amd64
# mv iperf_2.0.2-4_amd64 /usr/bin/iperf
\end{verbatim}

\subsection{参数说明及示例}

\begin{tabular}{lp{25em}}
  \toprule
  参数名       & 参数说明 \\
  \midrule
  --server	& 以服务端模式运行 \\
  --udp	        & 指定测试UDP,默认为TCP带宽测试 \\
  --client <host>	& 以客户端模式运行,并连接<host> \\
  --bandwidth	& 指定测试中所使用的带宽,单位为[KM],默认为1Mbit/sec \\
  --time	        & 指定测试时间,单位为秒 \\
  --interval	& 指定多少时间间隔来报告测试结果,时间单位为秒 \\
  --format [kmKM]	& 指定报告的输出格式,单位分别为Kbits,Mbits,Kbytes,Mbytes \\
\bottomrule
\end{tabular}

\begin{enumerate}[itemsep=0pt,parsep=0pt]
\item 以太网UDP丢包率测试

\begin{verbatim}
A0304010:~ # iperf --server --udp 
A0305010:~ # iperf --udp --client 172.16.25.39 --interval 1 \
             --time 120 --bandwidth 900M
\end{verbatim}

\item InfiniBand网络UDP丢包率测试

\begin{verbatim}
# iperf --server --udp 
# iperf --udp --client 11.11.11.39 --interval 1 \
             --time 120 --bandwidth 1024M
\end{verbatim}

\item 如果不指定--udp选项,默认就是测试TCP带宽

\begin{verbatim}
A0304010:~ # iperf --server 
A0305010:~ # iperf --client 172.16.25.39  -f M
\end{verbatim}
\end{enumerate}

\section{vmstat命令的使用}
\label{sec:vmstatCmd}

Linux下vmstat输出释疑:

\begin{verbatim}
Vmstat
procs -----------memory---------- ---swap-- -----io---- --system-- ----cpu----
r b   swpd free buff cache          si so      bi bo      in cs    us sy id wa
0 0   100152 2436 97200 289740       0 1       34 45       99 33    0 0 99 0
\end{verbatim}

\begin{quote}
procs
r 列表示运行和等待cpu时间片的进程数,如果长期大于cpu个数,说明cpu不足,需要增加cpu。

b 列表示在等待资源的进程数,比如正在等待I/O、或者内存交换等。

memory
swpd 切换到内存交换区的内存数量(k表示)。如果swpd的值不为0,或者比较大,比如超过了100m,
     只要si、so的值长期为0,系统性能还是正常

free 当前的空闲页面列表中内存数量(k表示)

buff 作为buffer cache的内存数量,一般对块设备的读写才需要缓冲。

cache: 作为page cache的内存数量,一般作为文件系统的cache,如果cache较大,说明用到cache的
       文件较多,如果此时IO中bi比较小,说明文件系统效率比较好。

swap
si 由内存进入内存交换区数量。

so 由内存交换区进入内存数量。


IO
bi 从块设备读入数据的总量(读磁盘)(每秒kb)。

bo 块设备写入数据的总量(写磁盘)(每秒kb)


system 显示采集间隔内发生的中断数

in 列表示在某一时间间隔中观测到的每秒设备中断数。

cs 列表示每秒产生的上下文切换次数,如当cs比磁盘I/O和网络信息包速率高得多,都应进行进一步调查。



cpu 表示cpu的使用状态

us 列显示了用户方式下所花费 CPU 时间的百分比。us的值比较高时,说明用户进程消耗的cpu时间多,
   但是如果长期大于50\%,需要考虑优化用户的程序。

sy 列显示了内核进程所花费的cpu时间的百分比。这里us + sy的参考值为80\%,如果us+sy 大于
   80\%说明可能存在CPU不足。

wa 列显示了IO等待所占用的CPU时间的百分比。这里wa的参考值为30\%,如果wa超过30\%,说明IO等待严重,
   这可能是磁盘大量随机访问造成的,也可能磁盘或者磁盘访问控制器的带宽瓶颈造成的(主要是块操作)。

id 列显示了cpu处在空闲状态的时间百分比
\end{quote}


\section{iostat命令的使用}
\label{sec:iostatCmd}

\section{sar命令的使用}
\label{sec:sarCmd}

sar 命令行的常用格式:
\begin{verbatim}
sar [options] [-A] [-o file] t [n]
\end{verbatim}

在命令行中,n 和t 两个参数组合起来定义采样间隔和次数,t为采样间隔,是必
须有的参数,n为采样次数,是可选的,默认值是1,-o file表示将命令结果以二
进制格式存放在文件中,file 在此处不是关键字,是文件名。options 为命令行
选项,sar命令的选项很多,下面只列出常用选项:

\begin{table}[!htbp]
  \centering
  \begin{tabular}{ll}
    \toprule
    选项     & 说明 \\
    \midrule
    -A  & 所有报告的总和 \\
    -u  & CPU利用率 \\
    -v  & 进程、I节点、文件和锁表状态 \\
    -d  & 硬盘使用报告 \\
    -r  & 没有使用的内存页面和硬盘块 \\
    -g  & 串口I/O的情况(centos 5 中无此选项) \\
    -b  & 缓冲区使用情况 \\
    -a  & 文件读写情况 \\
    -c  & 系统调用情况 \\
    -R  & 进程的活动情况 \\
    -y  & 终端设备活动情况 \\
    -w  & 系统交换活动 \\
    \bottomrule
  \end{tabular}
  \caption{sar常用选项}
  \label{tab:sarOptions}
\end{table}

例一:使用命令行 sar -u t n

例如,每60秒采样一次,连续采样5次,观察CPU 的使用情况,并将采样结果以二
进制形式存入当前目录下的文件lavenliu中,需键入如下命令:

\begin{verbatim}
# sar -u -o lavenliu 60 5
SCO_SV   scosysv 3.2v5.0.5 i80386   10/01/2001
14:43:50   %usr   %sys  %wio    %idle(-u)
14:44:50   0     1    4      94
14:45:50   0     2    4      93
14:46:50   0     2    2      96
14:47:50   0     2    5      93
14:48:50   0     2    2      96
Average    0     2    4      94
\end{verbatim}

在显示内容包括:

\begin{verbatim}
%usr:CPU处在用户模式下的时间百分比。
%sys:CPU处在系统模式下的时间百分比。
%wio:CPU等待输入输出完成时间的百分比。
%idle:CPU空闲时间百分比。
\end{verbatim}

在所有的显示中,我们应主要注意\%wio和\%idle,\%wio的值过高,表示硬盘存
在I/O瓶颈,\%idle值高,表示CPU较空闲,如果\%idle值高但系统响应慢时,有
可能是CPU等待分配内存,此时应加大内存容量。\%idle值如果持续低于10,那么
系统的CPU处理能力相对较低,表明系统中最需要解决的资源是CPU。

如果要查看二进制文件lavenliu中的内容,则需键入如下sar命令:

\begin{verbatim}
# sar -u -f lavenliu
\end{verbatim}

可见,sar命令即可以实时采样,又可以对以往的采样结果进行查询。

例二:使用命行sar -v t n

例如,每30秒采样一次,连续采样5次,观察核心表的状态,需键入如下命令:

\begin{verbatim}
# sar -v 30 5
SCO_SV scosysv 3.2v5.0.5 i80386 10/01/2001
10:33:23 proc-sz ov inod-sz ov file-sz ov lock-sz   (-v)
10:33:53  305/ 321  0 1337/2764  0 1561/1706 0 40/ 128
10:34:23  308/ 321  0 1340/2764  0 1587/1706 0 37/ 128
10:34:53 305/ 321  0 1332/2764  0 1565/1706 0 36/ 128
10:35:23 308/ 321  0 1338/2764  0 1592/1706 0 37/ 128
10:35:53 308/ 321  0 1335/2764  0 1591/1706 0 37/ 128
\end{verbatim}

显示内容包括:

\begin{quote}
proc-sz:目前核心中正在使用或分配的进程表的表项数,由核心参数MAX-PROC控制。
inod-sz:目前核心中正在使用或分配的i节点表的表项数,由核心参数MAX-INODE控制
file-sz:目前核心中正在使用或分配的文件表的表项数,由核心参数MAX-FILE控制。
ov:     溢出出现的次数。
Lock-sz:目前核心中正在使用或分配的记录加锁的表项数,由核心参数MAX-FLCKRE控制。
\end{quote}

显示格式为: 实际使用表项/可以使用的表项数

显示内容表示,核心使用完全正常,三个表没有出现溢出现象,核心参数不需调
整,如果出现溢出时,要调整相应的核心参数,将对应的表项数加大。

例三:使用命行sar -d t n

例如,每30秒采样一次,连续采样5次,报告设备使用情况,需键入如下命令:
\begin{verbatim}
# sar -d 30 5
SCO_SV scosysv 3.2v5.0.5 i80386 10/01/2001
11:06:43 device %busy   avque   r+w/s  blks/s  avwait avserv (-d)
11:07:13 wd-0   1.47   2.75   4.67   14.73   5.50 3.14
11:07:43 wd-0   0.43   18.77   3.07   8.66   25.11 1.41
11:08:13 wd-0   0.77   2.78   2.77   7.26   4.94 2.77
11:08:43 wd-0   1.10   11.18   4.10   11.26   27.32 2.68
11:09:13 wd-0   1.97   21.78   5.86   34.06   69.66 3.35
Average wd-0   1.15   12.11   4.09   15.19   31.12 2.80
\end{verbatim}

显示内容包括:
\begin{verbatim}
device: sar命令正在监视的块设备的名字。
%busy: 设备忙时,传送请求所占时间的百分比。
avque: 队列站满时,未完成请求数量的平均值。
r+w/s: 每秒传送到设备或从设备传出的数据量。
blks/s: 每秒传送的块数,每块512字节。
avwait: 队列占满时传送请求等待队列空闲的平均时间。
avserv: 完成传送请求所需平均时间(毫秒)。
\end{verbatim}

在显示的内容中,wd-0是硬盘的名字,\%busy的值比较小,说明用于处理传送请求
的有效时间太少,文件系统效率不高,一般来讲,\%busy值高些,avque值低些,
文件系统的效率比较高,如果\%busy和avque值相对比较高,说明硬盘传输速度太
慢,需调整。

例四:使用命行sar -b t n

例如,每30秒采样一次,连续采样5次,报告缓冲区的使用情况,需键入如下命令:
\begin{verbatim}
# sar -b 30 5
SCO_SV scosysv 3.2v5.0.5 i80386 10/01/2001
14:54:59 bread/s lread/s %rcache bwrit/s lwrit/s %wcache pread/s pwrit/s (-b)
14:55:29 0  147  100  5  21  78   0   0
14:55:59 0  186  100  5  25  79   0   0
14:56:29 4  232   98  8  58  86   0   0
14:56:59 0  125  100  5  23  76   0   0
14:57:29 0   89  100  4  12  66   0   0
Average  1  156   99  5  28  80   0   0
\end{verbatim}

显示内容包括:
\begin{verbatim}
bread/s: 每秒从硬盘读入系统缓冲区buffer的物理块数。
lread/s: 平均每秒从系统buffer读出的逻辑块数。
%rcache: 在buffer cache中进行逻辑读的百分比。
bwrit/s: 平均每秒从系统buffer向磁盘所写的物理块数。
lwrit/s: 平均每秒写到系统buffer逻辑块数。
%wcache: 在buffer cache中进行逻辑读的百分比。
pread/s: 平均每秒请求物理读的次数。
pwrit/s: 平均每秒请求物理写的次数。
\end{verbatim}

在显示的内容中,最重要的是\%cache和\%wcache两列,它们的值体现着buffer的
使用效率,\%rcache的值小于90或者\%wcache的值低于65,应适当增加系
统buffer的数量,buffer数量由核心参数NBUF控制,使\%rcache达到90左
右,\%wcache达到80左右。但buffer参数值的多少影响I/O效率,增加buffer,应
在较大内存的情况下,否则系统效率反而得不到提高。

例五:使用命行sar -g t n
例如,每30秒采样一次,连续采样5次,报告串口I/O的操作情况,需键入如下命令:
\begin{verbatim}
# sar -g 30 5
SCO_SV scosysv 3.2v5.0.5 i80386  11/22/2001
17:07:03  ovsiohw/s  ovsiodma/s  ovclist/s (-g)
17:07:33   0.00   0.00   0.00
17:08:03   0.00   0.00   0.00
17:08:33   0.00   0.00   0.00
17:09:03   0.00   0.00   0.00
17:09:33   0.00   0.00   0.00
Average    0.00   0.00   0.00
\end{verbatim}

显示内容包括:
\begin{verbatim}
ovsiohw/s:每秒在串口I/O硬件出现的溢出。
ovsiodma/s:每秒在串口I/O的直接输入输出通道高速缓存出现的溢出。
ovclist/s :每秒字符队列出现的溢出。
\end{verbatim}

在显示的内容中,每一列的值都是零,表明在采样时间内,系统中没有发生串口
I/O溢出现象。

sar命令的用法很多,有时判断一个问题,需要几个sar命令结合起来使用,比如,
怀疑CPU存在瓶颈,可用sar -u 和sar -q来看,怀疑I/O存在瓶颈,可用sar -b、
sar -u和sar-d来看。

\section{ip命令的使用}
\label{sec:ipCmd}

\subsection{显示IP信息}

同样,在我的SUSE 11sp2 64bit上运行也是同样的输出,输出内容略。以下几个
包是SUSE的OEM厂商给出的Bash最新的升级包。
