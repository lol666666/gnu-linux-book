\chapter{常用命令使用}
\label{sec:BasicCommand}

\section{命令行下的快捷键}
\label{shortcut}

经常在命令行下工作的同志们,可能用的最多的就是两个上下键,主要用来调出
历史命令;使用左右箭头使光标向后或向前移动以修改上次使用过的命令。其实
这样做效率并不是很高,有了快捷键可以让我们的效率有所提高,而且看起来还
更专业、更加Awesome、更加Geek。掌握了这些快捷键,我们可以做到手不离主键
盘区域,完全可以忽略掉键盘上的四个可爱的箭头。当我们熟练之后,会越发喜
欢这种方式。

\subsection{常用快捷键介绍}

下面介绍一些作者在命令行下经常使用的快捷键,这些快捷键在Emacs下面是有同
样的效果的,不信?你可以试试看。其实,Emacs是Gnu/Linux系统下的命令行编
辑器,通过/etc/profile或/etc/bashrc等文件都可以找到相关的设置。

\begin{enumerate}[itemsep=0pt,parsep=0pt]
	\item Ctrl+A\index{Ctrl+A}快捷键
  \begin{quote}
    这里的A可以理解为Head。当我们按下此组合键时,光标就从当前位置移到了
    命令行的起始位置。别只顾着看,动手试试!
  \end{quote}

\item Ctrl+B\index{Ctrl+B}快捷键
  \begin{quote}
    这里的B可以理解为Backward,向后的意思。有时在命令行上,我们把某个命
    令的参数或路径写错了,一般的做法是,使用左箭头,使光标移动到指定的
    位置,然后修改。其实我们完全可以使用Ctrl+B的方式以达到同样的效果。
    别只顾着看,动手试试!
  \end{quote}

\item Ctrl+C\index{Ctrl+C}快捷键
  \begin{quote}
    这个组合键是用来终止当前正在运行的前台进程。在UNIX环境高级编程一书
    上看到了一个用来终止当前运行进程的组合键,是Ctrl+\textbackslash
    \cite{unixenvironment}。别只顾着看,动手试试!
  \end{quote}

\item Ctrl+D\index{Ctrl+D}快捷键
  \begin{quote}
    这个组合键的用途也很广,我主要用此组合键来退出某个程序,如Python、
    MySQL等等。在命令行下意思就不同啦,此时的D可以理解为Delete。按下此
    组合键,会删除当前光标处的字符。别只顾着看,动手试试!
  \end{quote}

\item Ctrl+E\index{Ctrl+E}快捷键
  \begin{quote}
    这里的E可以理解为End。当在命令行按下此组合键时,我们的可爱的光标就
    乖乖地跑到了当前命令行的最后。\marginpar{这是边注一个}
  \end{quote}

\item Ctrl+F\index{Ctrl+F}快捷键
  \begin{quote}
    这里的F可以理解为Forward,向前的意思,等同于按下右箭头。别只顾着看,
    动手试试!
  \end{quote}

\item Ctrl+H\index{Ctrl+H}快捷键
  \begin{quote}
    此组合键相当于键盘上的Backspace键。按下此组合键,它会从当前光标处开
    始向后删除字符。别只顾着看,动手试试!
  \end{quote}

\item Ctrl+J\index{Ctrl+J}快捷键
  \begin{quote}
    此组合键相当于键盘的回车键。按下此组合键,相当于按了一次回车键。在
    Windows的命令行下,Ctrl+M好像是等同于回车键。别只顾看着,动手试试!
  \end{quote}

\item Ctrl+K\index{Ctrl+K}快捷键
  \begin{quote}
    这里的K可以理解为Kill。按下此组合键,会删除从当前光标到本命令行的结
    束的位置的所有字符。别只顾着看,动手试试!
  \end{quote}

\item Ctrl+L\index{Ctrl+L}快捷键
  \begin{quote}
    这里的L可以理解为Clear。按下此组合键相当于执行了clear这条命令,清除
    当前屏幕上的内容。别只顾着看,动手试试!
  \end{quote}

\item Ctrl+N\index{Ctrl+N}快捷键
  \begin{quote}
    这里的N可以理解为Next。这个组合键的作用是用来调出下一条历史命令,与
    之对应的快捷键Ctrl+P是调出上一条历史命令。代替了向下的箭头。别只顾
    着看,动手试试!
  \end{quote}

\item Ctrl+P\index{Ctrl+P}快捷键
  \begin{quote}
    这里的N可以理解为Previous。这个组合键的作用是用来调出上一条历史命令,
    与之对应的快捷键Ctrl+N是调出下一条历史命令。代替了向上的箭头。别只
    顾着看,动手试试!
  \end{quote}

\item Ctrl+R\index{Ctrl+R}快捷键
  \begin{quote}
    这个组合键是用来搜索之前的历史命令。这里的R可以理解为Reverse,反向
    的意思。在Emacs里为向后搜索,与之对应的是Ctrl+S快捷键是向前搜索。不
    过Ctrl+S在命令行里却不是这个作用,而是用来锁屏的。别只顾着看,动手
    试试!
  \end{quote}

\item Ctrl+T\index{Ctrl+T}快捷键
  \begin{quote}
    此组合键是交换两个相邻字符的位置。交换的是当前光标处字符及其当前光
    标前面的字符。比如我们不小心把clear命令写成了clera,此时我们也不用
    把ra两个字符删掉,然后再写上正确的。此时使我们的光标位于字符a上,让
    后按下此组合键,是不是神奇的事情发生了?当然,如果光标在行尾,按下
    此组合键,它会交换光标前的两个连续的字符。在Emacs下面,使用Ctrl+X与
    Ctrl+T两个组合键\footnote{先按下Ctrl+X,然后松开X,继续
      按着Ctrl键,然后再按下T键,即可完成两个组合键的操作。别嫌麻烦,习
      惯就好了。},可以交换当前光标行与上一行的位置。别只顾着看,动手试
    试!
  \end{quote}

\item Ctrl+W\index{Ctrl+W}快捷键
  \begin{quote}
    此组合键在Emacs中的作用是剪切选中区域的文本。在命令行上使用该组合键
    则是往后删除一个字符组合。也就是说,删除光标左边的一个字母组合或单
    词。比如,我们在此命令行上使用了命令如下,“service network
    restart”,让我们的光标位于字符串的restart的后面,按下该组合键,看看
    有何效果?是不是变成“service network”了?确实是这样,如果我们使用
    Backspace键的话,则需要使用7次的按键才能达到一个Ctrl+W的组合键的效
    果。嗯,别只顾着看,动手试试?
  \end{quote}

\item Alt+.\index{Alt+.}快捷键
  \begin{quote}
    此组合键是调出上一条命令的最后一个参数。如上一条命是“service
    network restart”,则“restart”就是最后一个参数。如果我们接下来要敲的
    命令需要用到上一条命令的最后一个参数,则可使用此快捷键,而不需要手
    工输入“restart”了,而且不会出错,节省敲击键盘的次数。如果我们接下来
    想重启httpd服务,则只需要输入“service httpd ”,然后按下“Alt+.”即可
    补全上一条命令的“restart”。在有些终端上,按“Alt+.”组合键可能会没有
    效果,这时可以使用“ESC+.”组合键代替。在Emacs中,ESC键与Alt键是等价
    的。可以动手试试该组合键的效果。
  \end{quote}

\end{enumerate}


\section{使用man page获得帮助}
\label{sec:getHelp}

当我们遇到不会用的系统命令时,该怎么办呢?或许你第一个想到的
是Baidu或Google,这样想很正常,可以节省很多时间。如果每次遇到不会的命令,
都去找网络,个人觉得这不是一件好的事情,不利于我们的提高。

如果不依靠互联网,该怎么解决呢?那就是依赖系统自带的man page\index{man
  page}了。通过它我们可以获取绝大部分的帮助信息,这正是锻炼我们的时候。当然我们也可以使用强大的info工具来查看帮助。

\section{echo与终端颜色}
\label{sec:echoCmd}

echo\index{echo}会将输入的字符串送往标准输出。输出的字符串间以空白字符隔开, 并在最
后加上换行号。

参数:

\begin{enumerate}[itemsep=0pt,parsep=0pt]
\item \-n 不要在最后自动换行 
\item \-e 若字符串中出现以下字符,则特别加以处理,而不会将它当成一般文字输出: 
\begin{verbatim}
\a 发出警告声; 
\b ***前一个字符; 
\c 最后不加上换行符号; 
\f 换行但光标仍旧停留在原来的位置; 
\n 换行且光标移至行首; 
\r 光标移至行首,但不换行; 
\t 插入tab; 
\v 与\f相同; 
\\ 插入\字符; 
\nnn 插入nnn(八进制)所代表的ASCII字符; 
–help 显示帮助 
–version 显示版本信息
\end{verbatim}
\end{enumerate}

\subsection{终端颜色}

echo字体颜色和背景颜色 

-e enable interpretation of the backslash-escaped characters listed below 

字背景颜色范围:40-47

\begin{table}[!htbp]
  \centering
  \begin{tabular}{llll}
    \toprule
    R & G & B & Color \\
    \midrule
    0 & 0 & 0 & Black \\
    0 & 0 & 1 & Blue \\
    0 & 1 & 0 & Green \\
    0 & 1 & 1 & Cyan \\
    1 & 0 & 0 & Red \\
    1 & 0 & 1 & Magenta \\
    1 & 1 & 0 & Yellow \\
    1 & 1 & 1 & White \\
    \bottomrule
  \end{tabular}
  \caption{颜色表\cite{computersystem}}
  \label{tab:colorTable}
\end{table}

\begin{verbatim}
40:黑 
41:深红 
42:绿 
43:*** 
44:蓝色 
45:紫色 
46:深绿 
47:白色
\end{verbatim}

字颜色:30-37

ANSI控制码的说明:

\begin{verbatim}
\e[0m 关闭所有属性 
\e[1m 设置高亮度 
\e[4m 下划线 
\e[5m 闪烁 
\e[7m 反显 
\e[8m 消隐 
\e[30m — \e[37m 设置前景色 
\e[40m — \e[47m 设置背景色 
\e[nA 光标上移n行 
\e[nB 光标下移n行 
\e[nC 光标右移n行 
\e[nD 光标左移n行 
\e[y;xH设置光标位置 
\e[2J 清屏 
\e[K 清除从光标到行尾的内容 
\e[s 保存光标位置 
\e[u 恢复光标位置 
\e[?25l 隐藏光标 
\e[?25h 显示光标
\end{verbatim}

下面看一个例子:
\begin{verbatim}
for i in `seq 0 7` ; do echo -e "\033[30;4${i}m      \033[0m"; \ 
done
\end{verbatim}

输出结果为:
\begin{figure}[hbtp]
  \centering
  \includegraphics[width=.15\textwidth]{img/color.png}
  \caption{终端颜色效果}
  \label{fig:TermColor}
\end{figure}

\section{date命令的使用}
\label{sec:dateCmd}

\index{date}

\small{
\begin{verbatim}
# 设日期
date -s 20091112                     

# 设时间
date -s 18:30:50                     

# 7天前日期
date -d "7 days ago" +%Y%m%d         

# 5分钟前时间
date -d "5 minute ago" +%H:%M        

# 一个月前
date -d "1 month ago" +%Y%m%d        

# 日期格式转换
date +%Y-%m-%d -d '20110902'         

# 日期和时间
date +%Y-%m-%d_%X                    

# 纳秒
date +%N                             

# 换算成秒计算(1970年至今的秒数)
date -d "2012-08-13 14:00:23" +%s    

# 将时间戳换算成日期
date -d "@1363867952" +%Y-%m-%d-%T   

# 将时间戳换算成日期
date -d "1970-01-01 UTC 1363867952 seconds" +%Y-%m-%d-%T  

# 格式化系统启动时间(多少秒前)
date -d "`awk -F. '{print $1}' /proc/uptime` second ago" +"%Y-%m-%d %H:%M:%S"    
\end{verbatim}
}
\normalsize

\input{body/chapter02_yum}

\input{body/chapter02_zypper}

\section{parted命令的使用}
\label{sec:PartedCmd}

Gnu/Linux系统的分区工具通常可以使用fdisk与parted。我们用的比较多的工具
就是fdisk了,这里不介绍它的使用了。这里简单的介绍如何使用parted工具,对
于分区表通常有MBR分区表和GPT分区表对于磁盘大小小于2T的磁盘,我们可以使
用fdisk和parted命令工具进行分区对于MBR分区表的特点(通常使用fdisk命令进
行分区)所支持的最大磁盘大小:2T最多支持4个主分区或者是3个主分区加上一
个扩展分区对于GPT分区表的特点(使用parted命令进行分区)支持最大
卷:18EB(1EB=1024TB)最多支持128个分区

对于parted命令工具分区的介绍

最后,fdisk与parted有些差异。fdisk分区完毕后,需要使用“w”命令才能保存
之前所做的一些操作;而parted则是实时的,每一步操作不需要保存,即时生
效。

\section{mount命令的使用}
\label{sec:mountCmd}

如何挂载iso镜像文件呢?我们可以使用一下mount\index{mount}命令:

\small{
\begin{verbatim}
[root@iLiuc ~]# mount -o loop rhel-server-5.5-i386-dvd.iso /mnt
意思是把挂rhel-server-5.5-i386-dvd.iso载到/mnt目录下,不过
你得事先有这个镜像文件
\end{verbatim}
}
\normalsize

\section{grep命令的使用}
\label{sec:grepCmd}

grep\index{grep}(global regular expression pattern的缩写)。其实可以把它
理解为过滤关键字用的一个程序。具体怎么用,还是看一个实例吧,然后结束本
节内容。

\subsection{常用选项}

\begin{table}[!htbp]
  \centering
  \caption{grep常用选项}
  \begin{tabular}{l|l}
    \hline
    -A NUM  & 打印出紧随匹配的行之后的下文NUM行 \\
    \hline
    -B NUM  & 打印出匹配的行之前的上文NUM行 \\
    \hline
    -C NUM  & 打印出匹配的行的上下文前后各NUM行 \\
    \hline
    -b      & 在输出的每行前面同时打印出当前行在输入文件中的字节偏移量 \\
    \hline
    -c      & 显示匹配的行数 \\
    \hline
    -f file & 从文件file中获取模式,每行一个 \\
    \hline
    -H      & 为每个匹配的文件打印文件名 \\
    \hline
    -I      & 不搜索二进制文件 \\
    \hline
    -i      & 忽略大小写 \\
    \hline
    -l      & 只显示有匹配的文件的文件名 \\
    \hline
    -L      & 只显示未匹配的文件的文件名 \\
    \hline
    -n      & 输出行号 \\
    \hline
    -o      & 只显示匹配字段 \\
    \hline
    -q      & quiet静默模式 \\
    \hline
    -v      & 只显示不匹配的行 \\
    \hline
  \end{tabular}
\end{table}

\subsection{一些实例}

去掉文件里的注释行和空白行
\begin{verbatim}
# cat filename | grep -v ^$ | grep -v ^# | sudo tee squid.conf
\end{verbatim}

这里我们使用grep命令及cut命令一起把eth0上的IP给取出来,看操作:

\begin{verbatim}
  [root@iLiuc ~]# ifconfig eth0
  eth0 Link encap:Ethernet  HWaddr 5A:B6:4E:85:55:44  
  inet addr:192.168.18.18  Bcast:192.168.18.255  Mask:255.255.255.0
  inet6 addr: fe80::58b6:4eff:fe85:5544/64 Scope:Link
  UP BROADCAST RUNNING MULuTICAST  MTU:1500  Metric:1
  RX packets:11655331 errors:0 dropped:0 overruns:0 frame:0
  TX packets:1074797 errors:0 dropped:0 overruns:0 carrier:0
  collisions:0 txqueuelen:1000 
  RX bytes:4001012028 (3.7 GiB)  TX bytes:628740073 (599.6 MiB)
  Interrupt:185

  # 使用grep命令,匹配关键字,缩小范围
  [root@iLiuc ~]# ifconfig eth0 |grep "inet addr"
  inet addr:192.168.18.18  Bcast:192.168.18.255  Mask:255.255.255.0

  # 使用cut命令,把范围再次缩小些
  [root@iLiuc ~]# ifconfig eth0 |grep "inet addr" | cut -d: -f2
  192.168.18.18  Bcast

  # 是不是快出来了,再使用cut一次,IP地址就出来了
  [root@iLiuc ~]# 自己写出来吧,我已经写得很多了!
\end{verbatim}


\input{body/chapter02_crontab}

\section{find命令的使用}

find\index{find}命令很强大,强大到可以写很多东西。这里就介绍如何简单的
使用。直接看例子吧:

\small{
\begin{verbatim}
# linux文件无创建时间
# Access 使用时间  
# Modify 内容修改时间  
# Change 状态改变时间(权限、属主)
# 时间默认以24小时为单位,当前时间到向前24小时为0天,向前48-72小时为2天
# -and 且 匹配两个条件 参数可以确定时间范围 -mtime +2 -and -mtime -4
# -or 或 匹配任意一个条件

# 按文件名查找
find /etc -name http        

# 查找某一类型文件
find . -type f               

# 按照文件权限查找
find / -perm                 

# 按照文件属主查找
find / -user                 

# 按照文件所属的组来查找文件
find / -group                

# 文件使用时间在N天以内
find / -atime -n             

# 文件使用时间在N天以前
find / -atime +n             

# 文件内容改变时间在N天以内
find / -mtime -n             

# 文件内容改变时间在N天以前
find / -mtime +n             

# 文件状态改变时间在N天前
find / -ctime +n             

# 文件状态改变时间在N天内
find / -ctime -n             

# 查找文件长度大于1M字节的文件
find / -size +1000000c -print 

# 按名字查找文件传递给-exec后命令
find /etc -name "passwd*" -exec grep "root" {} \; 

# 查找文件名,不取路径
find . -name 't*' -exec basename {} \;  

# 批量改名(查找err替换为ERR {}文件
find . -type f -name "err*" -exec  rename err ERR {} \; 

# 查找任意一个关键字
find 路径 -name *name1* -or -name *name2* 
\end{verbatim}
}
\normalsize



\section{top命令的使用}
\label{sec:topCmd}

top\index{top}命令可动态显示服务器的进程信息,用户可以通过按键来刷新当
前状态。别的不多说,给个例子看看:

\begin{verbatim}
Tasks: 202 total,   2 running, 199 sleeping,   0 stopped,   1 zombie
Cpu(s):  7.9%us,  1.9%sy,  0.0%ni, 89.5%id,  0.7%wa,  0.0%hi,  0.0%si,  0.0%st
Mem:   6003152k total,  1909420k used,  4093732k free,    73688k buffers
Swap:  6180860k total,        0k used,  6180860k free,   893544k cached

  PID USER      PR  NI  VIRT  RES  SHR S %CPU %MEM    TIME+  COMMAND                                                   
 3852 richard   20   0 1329m 127m  31m S   22  2.2   2:11.41 vlc                                                                 
 2891 richard    9 -11  418m 6988 4660 S    6  0.1   0:37.86 pulseaudio 
 2880 richard   20   0 1669m  90m  35m S    5  1.6   1:35.32 gnome-shell
 2452 root      20   0  303m  74m  61m S    4  1.3   1:23.90 Xorg
 3051 richard   20   0  872m 200m  52m S    1  3.4   1:32.25 firefox
   10 root      20   0     0    0    0 S    0  0.0   0:00.57 rcuos/2
   77 root      20   0     0    0    0 R    0  0.0   0:01.30 kworker/3:1
  909 root     -51   0     0    0    0 S    0  0.0   0:10.11 irq/48-iwlwifi
 3025 richard   20   0  581m  19m  11m S    0  0.3   0:01.72 gnome-terminal
 3942 richard   20   0 99.6m  15m 5028 S    0  0.3   0:02.35 python
 4025 richard   20   0 17460 1408  980 R    0  0.0   0:00.04 top
    1 root      20   0 24740 2620 1352 S    0  0.0   0:00.88 init
    2 root      20   0     0    0    0 S    0  0.0   0:00.00 kthreadd
    3 root      20   0     0    0    0 S    0  0.0   0:00.08 ksoftirqd/0
    5 root       0 -20     0    0    0 S    0  0.0   0:00.00 kworker/0:0H                                                                            
    6 root      20   0     0    0    0 S    0  0.0   0:01.68 kworker/u16:0
    7 root      20   0     0    0    0 S    0  0.0   0:01.60 rcu_sched
    8 root      20   0     0    0    0 S    0  0.0   0:01.24 rcuos/0
    9 root      20   0     0    0    0 S    0  0.0   0:00.45 rcuos/1
   11 root      20   0     0    0    0 S    0  0.0   0:00.28 rcuos/3
   12 root      20   0     0    0    0 S    0  0.0   0:00.00 rcuos/4
   13 root      20   0     0    0    0 S    0  0.0   0:00.00 rcuos/5
   14 root      20   0     0    0    0 S    0  0.0   0:00.00 rcuos/6
   15 root      20   0     0    0    0 S    0  0.0   0:00.00 rcuos/7
   16 root      20   0     0    0    0 S    0  0.0   0:00.00 rcu_bh
\end{verbatim}

\section{free命令的使用}
\label{sec:freeCmd}

\begin{verbatim}
buffer:缓存磁盘上的数据
cached:缓存的是将要写往磁盘中的数据

buffer=used-buffer-cached
cached=free+buffer+cached
\end{verbatim}

\subsection{常用选项}
\label{subsec:freeOptions}

下面介绍几个常用的cut命令的选项,

\begin{table}[htbp]
  \centering
    \caption{free常用选项}
    \label{tab:freeSomeOpts}
    \begin{tabular}{cl}
      \toprule
      选项     & 说明 \\
      \midrule
      -c        & 按照字符进行分割 \\
      -d        & 指定分割字段的分隔符,默认是tab \\
      -f        & 指定要显示的列 \\
      \bottomrule
    \end{tabular}
\end{table}

\subsection{一些实例}
\label{subsec:freeInstances}

\section{xargs命令的使用}
\label{sec:xargsCmd}
\index{xargs}

\section{tr命令的使用}
\label{sec:trCmd}
\index{tr}

\section{tar命令的使用}
\label{sec:tarCmd}

tar\index{tar}命令是一个打包与解包的一个工具,功能很强大。下面介绍一些常用选项及使
用示例。

通用选项:

\begin{table}[htbp]
  \centering
    \caption{tar通用选项}
    \label{tab:tarGeneralOpt}
    \begin{tabular}{cl}
      \toprule
      选项     & 说明 \\
      \midrule
      j        & 使用bzip2的压缩方式 \\
      t        & 列出压缩包里有哪些文件,并不解压 \\
      z        & 使用gzip的压缩方式 \\
      f        & 指定输出的结果文件。该选项是必选的,不管是压缩还是解压缩 \\
      p        & 保留文件的所有权限 \\
      v        & 压缩或解压缩时,查看其打包过程 \\
      \bottomrule
    \end{tabular}
\end{table}

压缩时用的选项:

\begin{table}[!htbp]
  \centering
    \caption{tar压缩选项}
    \label{tab:tarCompressOpt}
    \begin{tabular}{cl}
      \toprule
      选项     & 说明 \\
      \midrule
      c        & 打包时用的选项,选项c与x不能同时出现 \\
      \bottomrule
    \end{tabular}
\end{table}

解压缩用的选项:

\begin{table}[htbp]
  \centering
    \caption{tar解压缩选项}
    \label{tab:tarUncompressOpt}
    \begin{tabular}{cl}
      \toprule
      选项     & 说明 \\
      \midrule
      x        & 解包时用的选项,选项c与x不能同时出现 \\
      \bottomrule
    \end{tabular}
\end{table}

举例说明:

\small{
\begin{verbatim}
  [root@iLiuc ~]# ls
  360fy        CLEAR-VOD-INSTALLPACKGE.V.2.0.8.tar
  这里有两个文件,第一个为目录,第二个为压缩包

  1. 创建tar包,不压缩
  [root@iLiuc ~]# tar -cvf 360fy.tar 360fy
  360fy/
  360fy/clearVodMS_360fy.tar.gz
  360fy/vod_yuezizhongxin.tar.gz
  360fy/clear_360fy.sql
  上面的例子我们使用-v选项,使我们可以看到过程。其中360fy.tar是我们创建的tar包,是针对
  360fy这个目录的,后面可以是一个或多个文件。

  2. 创建tar包,以gzip方式压缩
  [root@iLiuc ~]# tar -czvf 360fy.tar.gz 360fy

  3. 创建tar包,以bzip2的方式压缩
  [root@iLiuc ~]# tar -cjvf 360fy.tar.bz2 360fy

  4. 查看压缩包的内容,并不解压缩
  [root@iLiuc ~]# tar -tf 360fy.tar
  360fy/
  360fy/clearVodMS_360fy.tar.gz
  360fy/vod_yuezizhongxin.tar.gz
  360fy/clear_360fy.sql

  5. 解压缩
  不管是tar包,还是以gzip或bz2压缩的方式,我们使用一下这条命令都是通用的
  [root@iLiuc ~]# tar -xf 360fy.tar
  [root@iLiuc ~]# tar -xf 360fy.tar.gz
  [root@iLiuc ~]# tar -xf 360fy.tar.bz2
  这里我们没有加-v选项,可以加上-v选项以看到解压过程
\end{verbatim}
}
\normalsize


\input{body/chapter02_read}

\section{cut命令的使用}
\label{sec:cutCmd}

cut\index{cut}命令有的时候很有用,比如要获得指定的以某个分隔符分割的列
时,它就可以做到。其实还有比它更强大的工具,如sed及awk等,这里并不介绍
它们,后续章节有介绍。这里就不提及了,跳过它们吧。下面直接看例子吧:

\subsection{常用选项}
\label{subsec:cutOptions}

下面介绍几个常用的cut命令的选项,

\begin{table}[htbp]
  \centering
    \caption{cut常用选项}
    \label{tab:cutSomeOpts}
    \begin{tabular}{cl}
      \toprule
      选项     & 说明 \\
      \midrule
      -c        & 按照字符进行分割 \\
      -d        & 指定分割字段的分隔符,默认是tab \\
      -f        & 指定要显示的列 \\
      \bottomrule
    \end{tabular}
\end{table}

\subsection{一些实例}
\label{subsec:cutInstances}

接下来演示cut命令的一些常用示例,比如我们要查看/etc/passwd文件中用户名,由于用户名是位于/etc/passwd文件中的每一行的第一个以冒号为分割的字段,因此,可以使用cut命令轻松实现取出用户名的需求,

\begin{verbatim}
# cut -d: -f1 /etc/passwd

-d 指定区分列的定界,默认是tab
-f 指定要显示的列
-c 按字符来切割,如
# cut -c1-6 file (取文件第一行的前6个字符)

实验:取IP地址
# ifconfig wlan0 | grep 'inet addr' | cut -d: -f2 | cut -d' ' -f1
# hostname -i[I]

# 显示系统中总内存量
# free |tr -s ' ' |sed '/^Mem/!d' |cut -d" " -f2
\end{verbatim}


\begin{verbatim}
[root@iLiuc ~]# cat test.txt
chuanchuan:goodboy
chuanchuan goodboy

# 以冒号为分割,显示第一列
[root@iLiuc ~]# cut -d: -f1 test.txt
chuanchuan
chuanchuan goodboy

# 以冒号为分割,显示第二列
[root@iLiuc ~]# cut -d: -f2 test.txt
goodboy
chuanchuan goodboy

# 以空格为分割,显示第一列
[root@iLiuc ~]# cut -d" " -f1 test.txt
chuanchuan:goodboy
chuanchuan
\end{verbatim}

我想,知道这么多,应该就可以了。


\input{body/chapter02_sort}

\section{lsof命令的使用}
\label{sec:lsofCmd}
\index{lsof}

lsof\index{lsof}是“list open files”的缩写,是一个列出当前系统打开文件的
工具。在linux环境下,任何事物都以文件的形式存在,通过文件不仅仅可以访问
常规数据,还可以访问网络连接和硬件。所以如传输控制协议 (TCP) 和用户数据
报协议(UDP) 套接字等,系统在后台都为该应用程序分配了一个文件描述符,无
论这个文件的本质如何,该文件描述符为应用程序与基础操作系统之间的交互提
供了通用接口。因为应用程序打开文件的描述符列表提供了大量关于这个应用程
序本身的信息,因此通过lsof工具能够查看这个列表对系统监测以及排错将是很
有帮助的。

lsof不加任何选项,默认输出所有活动进程打开的文件。

\begin{verbatim}
show all connections with -i

# lsof -i
COMMAND  PID USER   FD   TYPE DEVICE SIZE NODE NAME
dhcpcd  6061 root   4u   IPv4   4510       UDP *:bootpc
sshd    7703 root   3u   IPv6   6499       TCP *:ssh (LISTEN)
sshd    7892 root   3u   IPv6   6757       TCP 10.10.1.5:ssh->192.168.1.5:49901 (ESTABLISHED)

# Get only IPv6 traffic with -i 6
# lsof -i 6

# show only tcp connections (works the same for udp)
# lsof -iTCP

# show networking related to a given port using -i :port
# lsof -i :22

# show connections to a specific host using @host
# lsof -i@172.16.25.18

# show connections based on the host and the port using @host:port
# lsof -i@172.16.25.18:22

# Find listening ports
# find ports that are awaiting connections
# lsof -i -sTCP:LISTEN

# User Information

# We can also get information on various users and what they're doing
# on the system, including their activity on the network, their
# interactions with files, etc.

# show what a given user has open using -u
# lsof -u postfix

# show what all users are doing except a certain user using -u ^user
# lsof -u ^postfix

# Kill everything a given user is doing
# kill -9 `lsof -t -u postfix`
\end{verbatim}

\subsection{恢复删除的文件}

有一次做研发的一位同事,因为系统根目录空间不足,想释放磁盘空间,结果使
用find命令找到了单个文件大于1GB的文件,发现/var/log/messages和
/var/log/warn文件每个都是1.3GB。他的系统根目录空间只有50GB的容量,结果
把/var/log/messages日志文件及/var/log/warn日志文给删除了。删除之后,却
没有达到的目的,磁盘使用空间依然是100\%。

殊不知,当进程打开了某个文件时,只要该进程保持打开该文件,即使将其删除,
它依然存在于磁盘中。这意味着,进程并不知道文件已经被删除,它仍然可以向
打开该文件时提供给它的文件描述符进行读取和写入。除了该进程之外,这个文
件是不可见的,因为已经删除了其相应的目录条目。

拿/var/log/messages文件为例,看看如何在故意删除后如何找回来。我们先看一
下/var/log/messages文件是什么进程打开的。

\begin{verbatim}
[root@iLiuc ~]# lsof /var/log/messages 
COMMAND    PID USER   FD   TYPE DEVICE SIZE/OFF    NODE NAME
syslog-ng 1493 root   10w   REG    8,2 57599903 3080573 /var/log/messages
\end{verbatim}

该命令的输出表明,/var/log/messages文件由syslog-ng进程打开,当前的进程
号为1493,用户身份是root,打开的文件描述符为10且该文件处于只写模式,对
应的TYPE(类型)为REG(常规)文件、磁盘位置、文件大小、索引节点。下面一
一验证:

\begin{verbatim}
[root@iLiuc ~]# stat /var/log/messages 
  File: `/var/log/messages'
  Size: 57603503  	Blocks: 112632     IO Block: 4096   regular file
Device: 802h/2050d	Inode: 3080573     Links: 1
Access: (0640/-rw-r-----)  Uid: (    0/    root)   Gid: (    0/    root)
Access: 2015-03-12 15:33:28.000000000 +0800
Modify: 2015-03-12 16:02:01.000000000 +0800
Change: 2015-03-12 16:02:01.000000000 +0800
 Birth: -
\end{verbatim}

现在,我们可以删除该文件以模拟误删除,

\begin{verbatim}
[root@iLiuc ~]# rm -f /var/log/messages
[root@iLiuc ~]# lsof -n |grep '(deleted)'
syslog-ng 1493  root 10w  REG  8,2 57605375   3080573 /var/log/messages (deleted)
\end{verbatim}

文件已被删除,看怎么恢复吧!从上面的输出信息可以看出之前打开该文件的进
程号及文件描述符,有了这两个信息就足够了。接下来,我们去cat一下/proc目
录中相应的目录中的文件描述符,

\begin{verbatim}
[root@iLiuc ~]# cat /proc/1493/fd/10 |head
Oct  9 03:55:32 linux syslog-ng[5747]: syslog-ng starting up; version='2.0.9'
Oct  9 03:55:32 linux syslog-ng[5747]: syslog-ng starting up; version='2.0.9'
Oct  9 03:55:32 linux syslog-ng[5747]: syslog-ng starting up; version='2.0.9'
Oct  9 03:55:32 linux syslog-ng[5747]: syslog-ng starting up; version='2.0.9'
Oct  9 03:55:32 linux syslog-ng[5747]: syslog-ng starting up; version='2.0.9'
Oct  9 03:55:32 linux syslog-ng[5747]: syslog-ng starting up; version='2.0.9'
Oct  9 03:55:32 linux syslog-ng[5747]: syslog-ng starting up; version='2.0.9'
Oct  9 03:55:32 linux syslog-ng[5747]: syslog-ng starting up; version='2.0.9'
Oct  9 03:55:32 linux syslog-ng[5747]: syslog-ng starting up; version='2.0.9'
Oct  9 03:55:32 linux syslog-ng[5747]: syslog-ng starting up; version='2.0.9'
\end{verbatim}

通过文件描述符查看了相应的数据,那么就可以使用I/O重定向将其复制到文件中,
如cat /proc/1493/fd/10 > /tmp/messages。此时,可以中止该守护进程(这将
  删除 FD,从而删除相应的文件),将这个临时文件复制到所需的位置,然后重
新启动该守护进程。

\begin{verbatim}
[root@iLiuc ~]# cat /proc/1493/fd/10 > /tmp/messages
[root@iLiuc ~]# /etc/init.d/syslog stop
[root@iLiuc ~]# cp /tmp/messages /var/log/messages
[root@iLiuc ~]# /etc/init.d/syslog start
[root@iLiuc ~]# wc -l /var/log/messages 
452113 /var/log/messages

[root@iLiuc ~]# lsof /var/log/messages 
COMMAND    PID USER   FD   TYPE DEVICE SIZE/OFF    NODE NAME
syslog-ng 3819 root    4w   REG    8,2 57608271 3080449 /var/log/messages
\end{verbatim}

我们可以看到,已删除的/var/log/messages文件已回来了!对于许多应用程序,
尤其是日志文件和数据库,这种恢复删除文件的方法非常有用。


\section{netstat命令的使用}
\label{sec:netstatCmd}
\index{netstat}

\section{tcpdump命令的使用}
\label{sec:tcpdumpCmd}

tcpdump\index{tcpdump}是一款基于命令行的工具,可以通过不同的命令行选项
来改变其状态、捕获数据的数量及捕获数据的方法。tcpdump提供的丰富选项可以
使你很容易的改变程序的运行方式。

下面列举一部分比较常用的选项,

\begin{table}[!htbp]
  \centering
  \caption{tcpdump常用选项}
  \label{tab:tcpdumpOptions}
  \begin{tabular}{ll}
    \toprule
    选项     & 说明 \\
    \midrule
    i     & 指定侦听的网络接口 \\
    v     & 指定详细模式输出详细的报文信息 \\
    vv    & 指定非常详细的模式输出及非常详细的报文信息 \\
    vvv   & 指定更加详细的模式输出及更详细的报文信息 \\
    x     & 规定tcpdump以16进制数格式显示数据包 \\
    X     & 规定tcpdump以hex及ASCII格式显示输出 \\
    XX    & 同上,并显示以太网头部信息 \\
    n     & 在捕获过程中不需要向DNS查询IP地址(显示IP地址及端口号) \\
    F     & 从指定的文件中读取表达式 \\
    D     & 显示tcpdump可以侦听的网络接口列表 \\
    c     & 指定捕获多少数据包,然后停止捕获 \\
    w     & 把捕获到的信息写到一个文件中 \\
    s     & 设置捕获数据包的长度为length \\
    \bottomrule
  \end{tabular}
\end{table}

第一种是关于类型的关键字,主要包括host, net, port,例如host
210.27.38.1,指明210.27.38.1是一台主机,net 202.0.0.0指明202.0.0.0是一个
网络地址,port 23指明端口是23。如果没有指定类型,缺省的类型是host。

\small{
\begin{verbatim}
# 想要截获所有210.27.38.1的主机收到的和发出的所有的数据包:
# tcpdump host 210.27.38.1

# 对本机的udp 123端口进行监视,123为ntp的服务端口
# tcpdump udp port 123

# 如果想要获取主机210.27.38.1接收或发出的telnet包,使用如下命令:
# tcpdump tcp port 23 host 210.27.38.1

# 想要获取主机210.27.38.1和主机210.27.38.2或210.27.38.3的通信,使用命令:
# 在命令行中使用括号时,一定要转义
# tcpdump host 210.27.38.1 and \(210.27.38.2 or 210.27.38.3\)

# 如果想要获取主机210.27.38.1除了和主机210.27.38.2之外所有主机通信的ip包,则:
# tcpdump ip host 210.27.38.1 and ! 210.27.38.2

# tcpdump icmp
# tcpdump port 3306
# tcpdump src port 1025
# tcpdump dst port 389
# tcpdump src port 1025 and tcp
# tcpdump udp and src port 53
\end{verbatim}
}
\normalsize


\section{traceroute命令的使用}
\label{sec:tracerouteCmd}

\section{wget命令的使用}
\label{sec:wgetCmd}



\input{body/chapter02_screen}

\input{body/chapter02_iptables}

\section{qperf命令的使用}
\label{sec:qperfCmd}

我们在做网络服务器的时候,通常会很关心网络的带宽和延迟。因为我们的很多
协议都是request-response协议,延迟决定了最大的QPS,而带宽决定了最大的负
荷。 通常我们知道自己的网卡是什么型号,交换机什么型号,主机之间的物理距
离是多少,理论上是知道带宽和延迟是多少的。但是现实的情况是,真正的带宽
和延迟情况会有很多变数的,比如说网卡驱动,交换机跳数,丢包率,协议栈配
置,就实际速度而言,都很大的影响了数值的估算。 所以我们需要找到工具来实
际测量下。

SUSE11sp2发行版里面自带,方便安装,专业有效,能够针对TCP和RDMA进行带宽
和延迟的详细测试。

\begin{verbatim}
# zypper install -y qperf
\end{verbatim}

由于我们需要测试Infiniband的传输速率,在安装之前请先确认安装
了InfiniBand的相关包,比如librdmacm,libibverbs等。另外,也可以选择使用
源码包编译和安装qperf,但是需要注意,在安装之前也需要将infiniband相关的
包先安装上,否则RDMA的相关测试也将无法进行。

\begin{verbatim}
# zypper install -y librdmacm libibverbs
\end{verbatim}

\subsection{参数说明及示例}

qperf分为服务器端和客户端。客户端通过发送请求并获得响应来获得服务器端和
客户端之间的网络带宽以及延迟等信息。

\begin{tabular}{lp{20em}}
  \toprule
  参数名       & 参数说明 \\
  \midrule
  <server\_ip>	& 指定服务器的地址 \\
  time            & 指定网络测试时间。默认单位为秒,单位可以通过后缀为m,h,d指定为分钟,小时,天 \\
  conf	        & 测试输出中显示本地和远端服务器和操作系统配置 \\
  use\_bits\_per\_sec & 使用b(bit)而不是B(byte)来显示网络速度 \\
  precision 2	& 设置显示小数点后几位。这里设置为显示小数点后两位 \\
  verbose\_more	& 显示更详细的配置和状态信息 \\
  loop msg\_size:1:1025k:*2 	& loop表示对指定的指标值进行轮询。这里设置为对msg\_size轮询1,2,4,8…1024k,获得对应的测试结果,下次测试的指标值是上次测试指标值的*2倍 \\
  tcp\_bw	& 对tcp的带宽进行测试 \\
  tcp\_lat	& 对tcp的延迟进行测试 \\
  udp\_bw	& 对udp的带宽进行测试 \\
  udp\_lat	& 对udp的延迟进行测试 \\
  sdp\_bw	& 对sdp的延迟进行测试 \\
  sdp\_lat	& 对sdp的延迟进行测试 \\
\bottomrule
\end{tabular}

\section{iperf命令的使用}
\label{sec:iperfCmd}

iperf工具我们主要

首先到官网获取iperf工具,并把该工具放到合适的位置。
\begin{verbatim}
# wget https://iperf.fr/download/iperf_2.0.2/iperf_2.0.2-4_amd64
# chmod +x iperf_2.0.2-4_amd64
# mv iperf_2.0.2-4_amd64 /usr/bin/iperf
\end{verbatim}

\subsection{参数说明及示例}

\begin{tabular}{lp{25em}}
  \toprule
  参数名       & 参数说明 \\
  \midrule
  --server	& 以服务端模式运行 \\
  --udp	        & 指定测试UDP,默认为TCP带宽测试 \\
  --client <host>	& 以客户端模式运行,并连接<host> \\
  --bandwidth	& 指定测试中所使用的带宽,单位为[KM],默认为1Mbit/sec \\
  --time	        & 指定测试时间,单位为秒 \\
  --interval	& 指定多少时间间隔来报告测试结果,时间单位为秒 \\
  --format [kmKM]	& 指定报告的输出格式,单位分别为Kbits,Mbits,Kbytes,Mbytes \\
\bottomrule
\end{tabular}

\begin{enumerate}[itemsep=0pt,parsep=0pt]
\item 以太网UDP丢包率测试

\begin{verbatim}
A0304010:~ # iperf --server --udp 
A0305010:~ # iperf --udp --client 172.16.25.39 --interval 1 \
             --time 120 --bandwidth 900M
\end{verbatim}

\item InfiniBand网络UDP丢包率测试

\begin{verbatim}
# iperf --server --udp 
# iperf --udp --client 11.11.11.39 --interval 1 \
             --time 120 --bandwidth 1024M
\end{verbatim}

\item 如果不指定--udp选项,默认就是测试TCP带宽

\begin{verbatim}
A0304010:~ # iperf --server 
A0305010:~ # iperf --client 172.16.25.39  -f M
\end{verbatim}
\end{enumerate}

\section{vmstat命令的使用}
\label{sec:vmstatCmd}

Linux下vmstat输出释疑:

\begin{verbatim}
Vmstat
procs -----------memory---------- ---swap-- -----io---- --system-- ----cpu----
r b   swpd free buff cache          si so      bi bo      in cs    us sy id wa
0 0   100152 2436 97200 289740       0 1       34 45       99 33    0 0 99 0
\end{verbatim}

\begin{quote}
procs
r 列表示运行和等待cpu时间片的进程数,如果长期大于cpu个数,说明cpu不足,需要增加cpu。

b 列表示在等待资源的进程数,比如正在等待I/O、或者内存交换等。

memory
swpd 切换到内存交换区的内存数量(k表示)。如果swpd的值不为0,或者比较大,比如超过了100m,
     只要si、so的值长期为0,系统性能还是正常

free 当前的空闲页面列表中内存数量(k表示)

buff 作为buffer cache的内存数量,一般对块设备的读写才需要缓冲。

cache: 作为page cache的内存数量,一般作为文件系统的cache,如果cache较大,说明用到cache的
       文件较多,如果此时IO中bi比较小,说明文件系统效率比较好。

swap
si 由内存进入内存交换区数量。

so 由内存交换区进入内存数量。


IO
bi 从块设备读入数据的总量(读磁盘)(每秒kb)。

bo 块设备写入数据的总量(写磁盘)(每秒kb)


system 显示采集间隔内发生的中断数

in 列表示在某一时间间隔中观测到的每秒设备中断数。

cs 列表示每秒产生的上下文切换次数,如当cs比磁盘I/O和网络信息包速率高得多,都应进行进一步调查。



cpu 表示cpu的使用状态

us 列显示了用户方式下所花费 CPU 时间的百分比。us的值比较高时,说明用户进程消耗的cpu时间多,
   但是如果长期大于50\%,需要考虑优化用户的程序。

sy 列显示了内核进程所花费的cpu时间的百分比。这里us + sy的参考值为80\%,如果us+sy 大于
   80\%说明可能存在CPU不足。

wa 列显示了IO等待所占用的CPU时间的百分比。这里wa的参考值为30\%,如果wa超过30\%,说明IO等待严重,
   这可能是磁盘大量随机访问造成的,也可能磁盘或者磁盘访问控制器的带宽瓶颈造成的(主要是块操作)。

id 列显示了cpu处在空闲状态的时间百分比
\end{quote}


\section{iostat命令的使用}
\label{sec:iostatCmd}

\section{sar命令的使用}
\label{sec:sarCmd}

sar 命令行的常用格式:
\begin{verbatim}
sar [options] [-A] [-o file] t [n]
\end{verbatim}

在命令行中,n 和t 两个参数组合起来定义采样间隔和次数,t为采样间隔,是必
须有的参数,n为采样次数,是可选的,默认值是1,-o file表示将命令结果以二
进制格式存放在文件中,file 在此处不是关键字,是文件名。options 为命令行
选项,sar命令的选项很多,下面只列出常用选项:

\begin{table}[!htbp]
  \centering
  \begin{tabular}{ll}
    \toprule
    选项     & 说明 \\
    \midrule
    -A  & 所有报告的总和 \\
    -u  & CPU利用率 \\
    -v  & 进程、I节点、文件和锁表状态 \\
    -d  & 硬盘使用报告 \\
    -r  & 没有使用的内存页面和硬盘块 \\
    -g  & 串口I/O的情况(centos 5 中无此选项) \\
    -b  & 缓冲区使用情况 \\
    -a  & 文件读写情况 \\
    -c  & 系统调用情况 \\
    -R  & 进程的活动情况 \\
    -y  & 终端设备活动情况 \\
    -w  & 系统交换活动 \\
    \bottomrule
  \end{tabular}
  \caption{sar常用选项}
  \label{tab:sarOptions}
\end{table}

例一:使用命令行 sar -u t n

例如,每60秒采样一次,连续采样5次,观察CPU 的使用情况,并将采样结果以二
进制形式存入当前目录下的文件lavenliu中,需键入如下命令:

\begin{verbatim}
# sar -u -o lavenliu 60 5
SCO_SV   scosysv 3.2v5.0.5 i80386   10/01/2001
14:43:50   %usr   %sys  %wio    %idle(-u)
14:44:50   0     1    4      94
14:45:50   0     2    4      93
14:46:50   0     2    2      96
14:47:50   0     2    5      93
14:48:50   0     2    2      96
Average    0     2    4      94
\end{verbatim}

在显示内容包括:

\begin{verbatim}
%usr:CPU处在用户模式下的时间百分比。
%sys:CPU处在系统模式下的时间百分比。
%wio:CPU等待输入输出完成时间的百分比。
%idle:CPU空闲时间百分比。
\end{verbatim}

在所有的显示中,我们应主要注意\%wio和\%idle,\%wio的值过高,表示硬盘存
在I/O瓶颈,\%idle值高,表示CPU较空闲,如果\%idle值高但系统响应慢时,有
可能是CPU等待分配内存,此时应加大内存容量。\%idle值如果持续低于10,那么
系统的CPU处理能力相对较低,表明系统中最需要解决的资源是CPU。

如果要查看二进制文件lavenliu中的内容,则需键入如下sar命令:

\begin{verbatim}
# sar -u -f lavenliu
\end{verbatim}

可见,sar命令即可以实时采样,又可以对以往的采样结果进行查询。

例二:使用命行sar -v t n

例如,每30秒采样一次,连续采样5次,观察核心表的状态,需键入如下命令:

\begin{verbatim}
# sar -v 30 5
SCO_SV scosysv 3.2v5.0.5 i80386 10/01/2001
10:33:23 proc-sz ov inod-sz ov file-sz ov lock-sz   (-v)
10:33:53  305/ 321  0 1337/2764  0 1561/1706 0 40/ 128
10:34:23  308/ 321  0 1340/2764  0 1587/1706 0 37/ 128
10:34:53 305/ 321  0 1332/2764  0 1565/1706 0 36/ 128
10:35:23 308/ 321  0 1338/2764  0 1592/1706 0 37/ 128
10:35:53 308/ 321  0 1335/2764  0 1591/1706 0 37/ 128
\end{verbatim}

显示内容包括:

\begin{quote}
proc-sz:目前核心中正在使用或分配的进程表的表项数,由核心参数MAX-PROC控制。
inod-sz:目前核心中正在使用或分配的i节点表的表项数,由核心参数MAX-INODE控制
file-sz:目前核心中正在使用或分配的文件表的表项数,由核心参数MAX-FILE控制。
ov:     溢出出现的次数。
Lock-sz:目前核心中正在使用或分配的记录加锁的表项数,由核心参数MAX-FLCKRE控制。
\end{quote}

显示格式为: 实际使用表项/可以使用的表项数

显示内容表示,核心使用完全正常,三个表没有出现溢出现象,核心参数不需调
整,如果出现溢出时,要调整相应的核心参数,将对应的表项数加大。

例三:使用命行sar -d t n

例如,每30秒采样一次,连续采样5次,报告设备使用情况,需键入如下命令:
\begin{verbatim}
# sar -d 30 5
SCO_SV scosysv 3.2v5.0.5 i80386 10/01/2001
11:06:43 device %busy   avque   r+w/s  blks/s  avwait avserv (-d)
11:07:13 wd-0   1.47   2.75   4.67   14.73   5.50 3.14
11:07:43 wd-0   0.43   18.77   3.07   8.66   25.11 1.41
11:08:13 wd-0   0.77   2.78   2.77   7.26   4.94 2.77
11:08:43 wd-0   1.10   11.18   4.10   11.26   27.32 2.68
11:09:13 wd-0   1.97   21.78   5.86   34.06   69.66 3.35
Average wd-0   1.15   12.11   4.09   15.19   31.12 2.80
\end{verbatim}

显示内容包括:
\begin{verbatim}
device: sar命令正在监视的块设备的名字。
%busy: 设备忙时,传送请求所占时间的百分比。
avque: 队列站满时,未完成请求数量的平均值。
r+w/s: 每秒传送到设备或从设备传出的数据量。
blks/s: 每秒传送的块数,每块512字节。
avwait: 队列占满时传送请求等待队列空闲的平均时间。
avserv: 完成传送请求所需平均时间(毫秒)。
\end{verbatim}

在显示的内容中,wd-0是硬盘的名字,\%busy的值比较小,说明用于处理传送请求
的有效时间太少,文件系统效率不高,一般来讲,\%busy值高些,avque值低些,
文件系统的效率比较高,如果\%busy和avque值相对比较高,说明硬盘传输速度太
慢,需调整。

例四:使用命行sar -b t n

例如,每30秒采样一次,连续采样5次,报告缓冲区的使用情况,需键入如下命令:
\begin{verbatim}
# sar -b 30 5
SCO_SV scosysv 3.2v5.0.5 i80386 10/01/2001
14:54:59 bread/s lread/s %rcache bwrit/s lwrit/s %wcache pread/s pwrit/s (-b)
14:55:29 0  147  100  5  21  78   0   0
14:55:59 0  186  100  5  25  79   0   0
14:56:29 4  232   98  8  58  86   0   0
14:56:59 0  125  100  5  23  76   0   0
14:57:29 0   89  100  4  12  66   0   0
Average  1  156   99  5  28  80   0   0
\end{verbatim}

显示内容包括:
\begin{verbatim}
bread/s: 每秒从硬盘读入系统缓冲区buffer的物理块数。
lread/s: 平均每秒从系统buffer读出的逻辑块数。
%rcache: 在buffer cache中进行逻辑读的百分比。
bwrit/s: 平均每秒从系统buffer向磁盘所写的物理块数。
lwrit/s: 平均每秒写到系统buffer逻辑块数。
%wcache: 在buffer cache中进行逻辑读的百分比。
pread/s: 平均每秒请求物理读的次数。
pwrit/s: 平均每秒请求物理写的次数。
\end{verbatim}

在显示的内容中,最重要的是\%cache和\%wcache两列,它们的值体现着buffer的
使用效率,\%rcache的值小于90或者\%wcache的值低于65,应适当增加系
统buffer的数量,buffer数量由核心参数NBUF控制,使\%rcache达到90左
右,\%wcache达到80左右。但buffer参数值的多少影响I/O效率,增加buffer,应
在较大内存的情况下,否则系统效率反而得不到提高。

例五:使用命行sar -g t n
例如,每30秒采样一次,连续采样5次,报告串口I/O的操作情况,需键入如下命令:
\begin{verbatim}
# sar -g 30 5
SCO_SV scosysv 3.2v5.0.5 i80386  11/22/2001
17:07:03  ovsiohw/s  ovsiodma/s  ovclist/s (-g)
17:07:33   0.00   0.00   0.00
17:08:03   0.00   0.00   0.00
17:08:33   0.00   0.00   0.00
17:09:03   0.00   0.00   0.00
17:09:33   0.00   0.00   0.00
Average    0.00   0.00   0.00
\end{verbatim}

显示内容包括:
\begin{verbatim}
ovsiohw/s:每秒在串口I/O硬件出现的溢出。
ovsiodma/s:每秒在串口I/O的直接输入输出通道高速缓存出现的溢出。
ovclist/s :每秒字符队列出现的溢出。
\end{verbatim}

在显示的内容中,每一列的值都是零,表明在采样时间内,系统中没有发生串口
I/O溢出现象。

sar命令的用法很多,有时判断一个问题,需要几个sar命令结合起来使用,比如,
怀疑CPU存在瓶颈,可用sar -u 和sar -q来看,怀疑I/O存在瓶颈,可用sar -b、
sar -u和sar-d来看。

% \input{body/chapter02_ip}